%\documentclass[twocolumn]{article}
%\documentclass{ecctd01}
%\usepackage{cas_conf}
%%\documentclass[letterpaper]{ecctd01}
\documentclass{ecctd01}

\usepackage[dvips]{epsfig}
\usepackage{epsfig}
\usepackage{psfrag}
\usepackage{pifont}
\usepackage{amssymb}
\usepackage{amsmath}
\usepackage{rotate}
\usepackage{rotating}
\usepackage{color}
\usepackage{comment}
%%\usepackage{abstract}

\bibliographystyle{/home/jarocho/tex/noise_castour/ieeebib}

%
%	LINear ALGebra Macros
%
%  a) Matrix & vector alike manipulations
%
%

%%BUT firstly, a macro to set the first letter and word of
%%a paagraph in a bigger size than the rest of the text.
%% arg1=1stletter, arg2=rest_of_word
%% Example
%% /*****\HIS is a big T.
%%    *
%%    *
\def\PARAstart#1#2{\begingroup\def\par{\endgraf\endgroup\lineskiplimit=0pt}
    \setbox2=\hbox{\uppercase{#2} }\newdimen\tmpht \tmpht \ht2
    \advance\tmpht by \baselineskip\font\hhuge=cmr10 at \tmpht
    \setbox1=\hbox{{\hhuge #1}}
    \count7=\tmpht \count8=\ht1\divide\count8 by 1000 \divide\count7 by\count8
    \tmpht=.001\tmpht\multiply\tmpht by \count7\font\hhuge=cmr10 at \tmpht
    \setbox1=\hbox{{\hhuge #1}} \noindent \hangindent1.05\wd1
    \hangafter=-2 {\hskip-\hangindent \lower1\ht1\hbox{\raise1.0\ht2\copy1}%
    \kern-0\wd1}\copy2\lineskiplimit=-1000pt}


%
%	HORIZONTAL VECTORS	x1, x2, ... xn

%	x1, x2, .. xn
%
\newcommand{\hvec}
	   {\mbox{$x_1, x_2,\cdots,x_n$ }}

%	(x)1, (x)2, ... (x)n
%
\newcommand{\hvecx}[1]
	   {\mbox{${#1}_1, {#1}_2,\cdots,{#1}_n$ }}

%
%	(x)1, (y)2, ... (z)n
\newcommand{\hvecxyz}[3]
	   {\mbox{${#1}_1, {#2}_2,\cdots,{#3}_n$ }}

%
%	VERTICAL VECTORS

\newcommand{\vvec}
		{\mbox{$
			\begin{array}{c}
			x_1 \\ x_2 \\ \vdots \\ x_n \\
			\end{array}
		$}}

%
\newcommand{\vvecx}[1]
		{\mbox{$
			\begin{array}{c}
			#1_1 \\ #1_2 \\ \vdots \\ #1_n \\
			\end{array}
		$}}

%
\newcommand{\vvecxyz}[3]
		{\mbox{$
			\begin{array}{c}
			#1_1 \\ #2_2 \\ \vdots \\ #3_n \\
			\end{array}
		$}}

%
\newcommand{\dotcol}
{\mbox{$ 
	\begin{array}{c}
		\cdots\vspace{3mm}\\
		\cdots\vspace{3mm}\\
		\cdots\vspace{3mm}\\
		\cdots\vspace{3mm}\\
	\end{array}
$}}


%
%	GENERIC MATRICES


%
\newcommand{\genmtrx}[3]
{\mbox{$	%\left[
\pmatrix{
	#1_{11}	   & #1_{12}	& \cdots & #1_{1{#2}}	\cr 
	#1_{21}	   & #1_{22}	& \cdots & #1_{2{#2}}	\cr 
	\vdots 	   & \vdots 	& \ddots & \vdots	\cr
	#1_{{#3}1} & #1_{{#3}2} & \cdots & #1_{{#3}{#2}}\cr 
		%\rigth]
}
	$}}

%

%
				%% [ p1_{p2 p3} ]
\newcommand{\biggmtrx}[3]
{\mbox{$
	\left[ 
	\begin{array}{c}
		{#1}_{11}	     \vspace{3mm}\\ 
		{#1}_{21}	     \vspace{3mm}\\
		\vdots	     \vspace{3mm}\\ 
		{#1}_{{#3}1}   \vspace{3mm}\\ 
	\end{array}
	\begin{array}{c}
		{#1}_{12}      \vspace{3mm}\\ 
		{#1}_{22}      \vspace{3mm}\\
		\vdots       \vspace{3mm}\\ 
		{#1}_{{#3}2}   \vspace{3mm}\\ 
	\end{array}
	\begin{array}{c}
		\cdots\vspace{3mm}\\
		\cdots\vspace{3mm}\\
		\ddots\vspace{3mm}\\
		\cdots
	\end{array}
	\begin{array}{c}
		{#1}_{1{#2}}         \vspace{3mm}\\ 
		{#1}_{2{#2}}         \vspace{3mm}\\
		\vdots            \vspace{3mm}\\
		{#1}_{{#3}{#2}}     \vspace{3mm}\\
	\end{array}
	\right]
$}}



%
% List of macros defining
% Some Jacobians and matrices used for numerical
% items.
%


			% | d p1_p2 |
\newcommand{\dpart}[4]	%$|---------|$
			% | d p3_p4 |
{\mbox{$
	\frac{\partial {#1}_{#2} }
	     {\partial {#3}_{#4} }
      $}
}

\newcommand{\bigjac}[2]
{\mbox{$
	\left[ 
	\begin{array}{c}
		\dpart{#1}{1}{#2}{1} \vspace{3mm}\\ 
		\dpart{#1}{2}{#2}{1} \vspace{3mm}\\
		\vdots \vspace{3mm}\\ 
		\dpart{#1}{n}{#2}{1} \vspace{3mm}\\ 
	\end{array}
	\begin{array}{c}
		\dpart{#1}{1}{#2}{2} \vspace{3mm}\\ 
		\dpart{#1}{2}{#2}{2} \vspace{3mm}\\
		\vdots \vspace{3mm}\\
		\dpart{#1}{n}{#2}{2} \vspace{3mm}\\ 
	\end{array}
	\begin{array}{c}
		\cdots\vspace{3mm}\\
		\cdots\vspace{3mm}\\
		\ddots\vspace{3mm}\\
		\cdots
	\end{array}
	\begin{array}{c}
		\dpart{#1}{1}{#2}{n} \vspace{3mm}\\ 
		\dpart{#1}{2}{#2}{n} \vspace{3mm}\\
		\vdots \vspace{3mm} \\
		\dpart{#1}{n}{#2}{n} \vspace{3mm}\\
	\end{array}
	\right]
$}}


\newcommand{\chuajac}[3]
{\mbox{$
	\left[ 
	\begin{array}{c}
		\dpart{#1}{1}{#2}{1} \vspace{3mm}\\ 
		\dpart{#1}{2}{#2}{1} \vspace{3mm}\\
		\vdots \vspace{3mm}\\ 
		\dpart{#1}{n}{#2}{1} \vspace{3mm}\\ 
		0
	\end{array}
	\begin{array}{c}
		\dpart{#1}{1}{#2}{2} \vspace{3mm}\\ 
		\dpart{#1}{2}{#2}{2} \vspace{3mm}\\
		\vdots \vspace{3mm}\\
		\dpart{#1}{n}{#2}{2} \vspace{3mm}\\ 
		0
	\end{array}
	\begin{array}{c}
		\cdots\vspace{3mm}\\
		\cdots\vspace{3mm}\\
		\vdots\vspace{3mm}\\
		\cdots\vspace{3mm}\\
		\cdots
	\end{array}
	\begin{array}{c}
		\dpart{#1}{1}{#2}{k} \vspace{3mm}\\ 
		\dpart{#1}{2}{#2}{k} \vspace{3mm}\\
		\vdots \vspace{3mm} \\
		\dpart{#1}{n}{#2}{k} \vspace{3mm}\\
		sgn(\frac{dx_k}{dp})
	\end{array}
	\begin{array}{c}
		\cdots\vspace{3mm}\\
		\cdots\vspace{3mm}\\
		\ddots\vspace{3mm}\\
		\cdots\vspace{3mm}\\
		\cdots
	\end{array}
	\begin{array}{c}
		\dpart{#1}{1}{#2}{n} \vspace{3mm}\\ 
		\dpart{#1}{2}{#2}{n} \vspace{3mm}\\
		\vdots \vspace{3mm} \\
		\dpart{#1}{n}{#2}{n} \vspace{3mm}\\
		0
	\end{array}
	\begin{array}{c}
		\dpart{#1}{1}{#3}{} \vspace{3mm}\\ 
		\dpart{#1}{2}{#3}{} \vspace{3mm}\\
		\vdots \vspace{3mm}\\
		\dpart{#1}{n}{#3}{} \vspace{3mm}\\
		0
	\end{array}
	\right]
$}}
%==============
%==============

\newcommand{\jacob}[2]
			{\mbox{$ \left[
			{\mbox{$ 
				{\coljac}{#1}{#2}{1}
			$}}
			{\mbox{$ 
				{\coljac}{#1}{#2}{2}
			$}}
			{\mbox{$ 
				{\dotcol}
			$}}
			{\mbox{$ 
				{\coljac}{#1}{#2}{n}
			$}}

			\rigth] $}}



\newcommand{\jac}[2]
{\mbox{$	%\left[
\pmatrix{
{\frac{\partial #1_1}{\partial #2_1} }& 
	{\frac{\partial #1_1}{\partial #2_2} }& 
		\cdots &
			{\frac{\partial #1_1}{\partial #2_n} }\cr 
{\frac{\partial #1_2}{\partial #2_1} }& 
	{\frac{\partial #1_2}{\partial #2_2} }& 
		\cdots &
			{\frac{\partial #1_2}{\partial #2_n} }\cr 
\vdots& \vdots& \ddots& \vdots\cr
{\frac{\partial #1_n}{\partial #2_1} }& 
	{\frac{\partial #1_n}{\partial #2_2} }& 
		\cdots &
			{\frac{\partial #1_n}{\partial #2_n} }\cr
		%\rigth]
}
	$}}


\newcommand{\coljac}[3]
{\mbox{$ 
	\begin{array}{c}
		\frac{\partial #1_1}{\partial #2_#3}\vspace{3mm} \\ 
		\frac{\partial #1_2}{\partial #2_#3}\vspace{3mm} \\ 
		\vdots \vspace{3mm} \\
		\frac{\partial #1_n}{\partial #2_#3}\vspace{3mm} \\ 
	\end{array}
$}}

%
%	C and O subset alike macros
% \cmd = normal comand
% \mcmd = comand surrounded by $ ... $
%

%
% C(T, B)
\newcommand{\cset}{\cal{C}(\pig{T}, \pig{B})}
\newcommand{\oset}{\cal{O}(\pig{T}, \pig{B})}
\newcommand{\mcset}{\mbox{$\cal{C}(\pig{T}, \pig{B})$} }
\newcommand{\moset}{\mbox{$\cal{O}(\pig{T}, \pig{B})$} }

\newcommand{\obe}{{\cal{O}}^{be}}
\newcommand{\mobe}{\mbox{${\cal{O}}^{be}$ }}
\newcommand{\omix}{{\cal{O}}^{mix}}
\newcommand{\momix}{\mbox{${\cal{O}}^{mix}$ }}
\newcommand{\obc}{{\cal{O}}^{bc}}
\newcommand{\mobc}{\mbox{${\cal{O}}^{bc}$ }}

%\newcommand{\pig}[1]{\mbox{\boldmath ${#1}$}	}



\newtheorem{Theod}{{\bf Definition}}
\newtheorem{Theos}{{\bf Statement}}
\newtheorem{Theor}{{\bf Theorem}}
\newtheorem{Theoc}{{\bf Corallary}}

%\setlength{\oddsidemargin}{0mm}
%\setlength{\evensidemargin}{0mm}
\setlength{\topmargin}{0mm}
\setlength{\textwidth}{6.5in}
%\setlength{\columnsep}{5mm}
\setlength{\textheight}{9.0in}
%\pagestyle{headings} %%Encabezados

\title{
Hierarchy in noise expressions for nullor-based amplifiers
}


\author{
Arturo Sarmiento-Reyes, Luis Hern\'andez-Mart\'{\i}nez,
Roberto Casta\~neda Sheissa,\\
Pr\'ospero L\'opez-S\'anchez\\
INAOE, Electronics Department, CAD Group, \\
P.O. Box 51 and 216, Puebla, Pue. 72000, Mexico
}

\begin{document}
%\maketitle

\maketitle

\begin{abstract}
In the structured design of high-performance negative-feedback amplifiers
the nullor is the basic block for the synthesis procedure.
The nullor represents the ideal infinite gain, i.e. the plant to be
controlled by the feedback network which is usually constituted by
passive components.
The paper establishes the hierarchical structure of the expressions for
the equivalent noise source at the input for all four types of single-loop
nullor-based amplifiers.
\end{abstract}

\section{Introduction}
In nullor-based amplifiers, two main blocks can be distinguished, the
feedback network and the active part.
As a result, the design task can be divided into two main steps
\cite{bias1, bias2,arizona}:
(i) the design of the feedback network, under the assumption that the
nullor constitutes the active (ideal) part, and
(ii) the synthesis of the nullor.
The whole design procedure is accomplished with the aim of optimising
three fundamental signal-processing quality aspects: noise, bandwidth
and distortion.

This article expounds the expressions for the equivalent input noise
source of the four types of nullor-based amplifiers and shows the inherent
hierarchy of the expressions, which can be used as a key-factor for
the design when optimising the amplifier vs noise.

\begin{figure}[hbtp]
\centerline{
\epsfxsize=70mm
\epsffile{/home/jarocho/tex/noise_castour/figuras/fig_nullor.eps}}
\caption{The nullor.}
\label{figure1}
\end{figure} 

\section{Nullor-based amplifiers}
The nullor constitutes the active (ideal) block of the amplifier.
In fact, the nullor is a two-port
composed of two elements:
{\bfseries \sffamily the nullator} connected at the input port
and
{\bfseries \sffamily the norator} connected at the output port
as depicted in Figure \ref{figure1}.

On the one hand, the nullator has the relationships:
\begin{displaymath}
u_i=0 \qquad i_i=0
\end{displaymath}

On the other hand, the norator has the relationships:
\begin{displaymath}
u_o= \mbox{arbitrary} \qquad i_o= \mbox{arbitrary}
\end{displaymath}
i.e. current and voltage of the output are defined by the
environment.

The transmission matrix of the nullor is given as \cite{carlin,moschytz}:
\begin{equation}\label{eq:abcd}
\pig{K}
=
\left [ \begin{array}{cc}
\frac{1}{\mu}& \frac{1}{\gamma} \\\\
\frac{1}{\zeta}& \frac{1}{\beta}
\end{array}
\right ]
=
\left [ \begin{array}{cc}
0 & 0 \\ 0 & 0
\end{array}
\right ]
\end{equation}
It clearly results that the nullor possesses infinite gains for all four
transfer relationships, voltage ($\mu$), current ($\alpha$),
trans-conductance ($\gamma$) and trans-impedance ($\zeta$).

\begin{figure}[hbtp]
\centerline{
\epsfxsize=78mm
\epsffile{/home/jarocho/tex/noise_castour/figuras/grupo_eng.eps}}
\caption{Negative-feedback amplifiers.}
\label{figure2}
\end{figure} 

The four types of amplifiers can be obtained when a passive feedback
network is connected with the nullor, in the negative-feedback
configurations shown in
Figure \ref{figure2}.
For each amplifier the feedback network determines the transfer function,
and then their ABCD matrix has only one non-zero element as shown
in equations (\ref{eq:mabcd1}),(\ref{eq:mabcd2}),(\ref{eq:mabcd3}),
(\ref{eq:mabcd4})

\begin{eqnarray}
\pig{K}_{u}=
\left[\begin{array}{cc}
\frac{1}{1+R_1/R_2}&0\\\\
0&0
\end{array} \right]
\label{eq:mabcd1}
\\
\pig{K}_{g_m} =
\left[\begin{array}{cc}
0&-\frac{1}{G}\\\\
0&0
\end{array}\right]
\label{eq:mabcd2}
\\
\pig{K}_{r_m} =
\left[\begin{array}{cc}
0&0\\\\
\frac{-1}{R}&0
\end{array}\right]
\label{eq:mabcd3}
\\
\pig{K}_{i} =
\left[\begin{array}{cc}
0&0\\\\
0&\frac{1}{1+R_2/R_1}
\end{array}\right]
\label{eq:mabcd4}
\end{eqnarray}

\subsection{Noise in the amplifier}
Noise can be regarded in terms of power spectral densities originated
from the passive network and the synthesised nullor.
Figure \ref{figure3} shows the four basic configurations of the noisy
amplifiers.
Herein, $u_{n,n}$ and $i_{n,n}$ model completely the noise present in 
the nullor as in any
two-port system \cite{haus,bias1}\footnote{Despite that
the two-port system is a nullor, the two noisy sources are
considered for the analysis because they appear later as
a result of the nullor synthesis.}.

\begin{figure}[hbtp]
\centerline{
\epsfxsize=78mm
\epsffile{/home/jarocho/tex/noise_castour/figuras/grupo_noise.eps}}
\caption{Noisy Negative-Feedback Amplifiers.}
\label{figure3}
\end{figure} 

Because, it is necessary to obtain the total equivalent noise 
(either current or voltage) at the input, the noise sources present in
the configurations from Figure \ref{figure3} must be moved from their
places to the input of the amplifier, and the rest of the amplifier
becomes noiseless. This is accomplished by resorting to
some simple transformations.

\section{Source transformations}
Several source movements or transformations are used in order to
obtain the total noise equivalent at the input.

\begin{itemize}
\item {\it V-Shift Transformation}\par
The V-shift on a voltage source $u_n$ is accomplished by moving the source
through the node \hspace{1.5ex}\encircle{2} and creating two new instances the sources
as shown in Figure {\ref{v-shift}}. After the transformation
is performed, the nodes \hspace{1.5ex}\encircle{1} and
\hspace{1.5ex}\encircle{2} are clustered.
\end{itemize}

\begin{figure}[hbtp]
\vspace{-2mm}
\centerline{
\epsfxsize=30mm
\epsffile{/home/jarocho/tex/noise_castour/figuras/v_shift.eps}}
\caption{V-Shift Transformation.}
\label{v-shift}
\end{figure} 

\begin{itemize}
\item {\it I-Shift.Transformation}\par
The I-shift transformation on a current source $i$ is obtained by
splitting it into
two ocurrence of $i_n$, as shown if Figure {\ref{i-shift}}.
\end{itemize}

\begin{figure}[hbtp]
\vspace{-2mm}
\centerline{
\epsfxsize=25mm
\epsffile{/home/jarocho/tex/noise_castour/figuras/i_shift.eps}}
\caption{I-Shift Transformation.}
\vspace{-5mm}
\label{i-shift}
\end{figure} 

\begin{itemize}
\item {\it Norton-Thevenin Transformation}\par
The theorems of Thevenin and Norton are used to transform voltage
sources into current sources and viceversa.
\end{itemize}


\begin{itemize}
\item {\it Two-port shift Transformation}\par
Figure {\ref{2port}} illustrates the shift of sources from the output
to the input of a two-port network charaterised by its $\pig{K}$
matrix.
\end{itemize} 

\begin{figure}[hbtp]
\vspace{-2mm}
\centerline{
\epsfxsize=78mm
\epsffile{/home/jarocho/tex/noise_castour/figuras/two_port_trans.eps}}
\caption{Two-Port Shift.}
\label{2port}
\end{figure} 

The transformations above, when applied on the configurations in Figure
\ref{figure3}, generate the following expressions for equivalent noise
at the input:\vspace{3.0ex}
\large
\begin{eqnarray}
\nonumber
\mbox{\fbox{\bfseries \sffamily Voltage amplifier}} & & \\\nonumber
u_{n,eq,in}= & 
\mbox{\textcolor{green}{$u_{n_{source}}$}} +
\mbox{\textcolor{red}{$u_{n_{nullor}}$}} +
\mbox{\textcolor{magenta}{$(R_s+\frac{R_1R_2}{R_1+R_2})i_{n_{nullor}}$}}
\qquad \qquad \qquad
&
\\\nonumber
&
\qquad \qquad \qquad \qquad \qquad \qquad
+\mbox{\textcolor{blue}{$\frac{\frac{R_1R_2}{R_1+R_2}}{R_1}u_{n_{R_1}}$}}
+\mbox{\textcolor{blue}{$\frac{\frac{R_1R_2}{R_1+R_2}}{R_2}u_{n_{R_2}}$}}
&
\\\label{vnoise}
\mbox{\fbox{\bfseries \sffamily Transconductance amplifier}} &&\\\nonumber
u_{n,eq,in}= &
\mbox{\textcolor{green}{$u_{n_{source}}$}} +
\mbox{\textcolor{red}{$u_{n_{nullor}}$}} +
\mbox{\textcolor{magenta}{$(R_s+R)i_{n_{nullor}}$}}
\qquad \qquad \qquad
&
\\\nonumber
&
\qquad \qquad \qquad \qquad \qquad \qquad
+ \mbox{\textcolor{blue}{$R i_{n_{R}}$}}
&
\\\label{gnoise}
\mbox{\fbox{\bfseries \sffamily Transimpedance amplifier}}\\\nonumber
i_{n,eq,in}= &
\mbox{\textcolor{green}{$i_{n_{source}}$}} +
\mbox{\textcolor{red}{$i_{n_{nullor}}$}} +
\mbox{\textcolor{magenta}{$(\frac{1}{R_s}+\frac{1}{R})u_{n_{nullor}}$}}
\qquad \qquad \qquad
&
\\\nonumber
&
\qquad \qquad \qquad \qquad \qquad \qquad
+ \mbox{\textcolor{blue}{$\frac{1}{R}u_{n_{R}}$}}
&
\\\label{rnoise}
\mbox{\fbox{\bfseries \sffamily Current amplifier}}\\\nonumber
i_{n,eq,in}= &
\mbox{\textcolor{green}{$i_{n_{source}}$}}+
\mbox{\textcolor{red}{$i_{n_{nullor}}$}} +
\mbox{\textcolor{magenta}{$(\frac{1}{R_s}+\frac{1}{R_1+R_2})u_{n_{nullor}}$}}
\qquad \qquad \qquad
&
\\\nonumber
&
\qquad \qquad \qquad \qquad \qquad \qquad
+\mbox{\textcolor{blue}{$\frac{R2}{R_1+R_2}i_{n_{R_2}}$}}
+\mbox{\textcolor{blue}{$\frac{R_1}{R_1+R_2}i_{n_{R_1}}$}}
&
\\\label{inoise}
\end{eqnarray}
\normalsize


\section{Hierarchy of the expressions}
The expressions (\ref{vnoise}--\ref{inoise}) for the equivalent input
noise for all amplifiers show a very uniform format.
\par
The variable in which the noise equivalent is expressed (the left hand side
of the expressions), depends on the type of
\underline{input} variable of the
amplifier. For instance, for the case of
a transconductance amplifier ($i_{out} = g_m u_{in}$) the input variable
is the voltage ($u_{in}$), and so is the variable for noise equivalent
($u_{n,eq,in}$).
\par
As a result, the terms of the right hand side of the expressions must
also have the same units of the input noise equivalent variable.
Several noise contributions may be identified on these terms:\vspace{-1.0ex}

\begin{itemize}
\item The noise from the source.\par
This contribution is of the same type of the input variable of the
amplifier. This is a data usually defined by the specs of the
source resistor, and therefore it is a figure that cannot be modified.
It is denoted either as $u_{n_{source}}$ or as $i_{n_{source}}$.
\item The nullor noise of the proper-variable.\par
This is the noise of the synthesised nullor expressed in the same
type of noise source of the input variable.
It is either $u_{n_{nullor}}$ or $i_{n_{nullor}}$.
$u_{n_{nullor}}$ is connected in series and
$i_{n_{nullor}}$ in parallel.
\end{itemize}
\par
\hrulefill\vspace{10.0cm}
\par
\hrulefill
\begin{itemize}
\item The nullor noise of the disjoint-variable.\par
The noise contribution of the synthesised nullor expressed as an
Ohm's Law relationship:\vspace{-0.5ex}
\begin{displaymath}
u = (R_s + R_{eq}) i_{n_{nullor}}
\end{displaymath}
or:\vspace{-0.5ex}
\begin{displaymath}
i = (G_s + G_{eq}) u_{n_{nullor}}\vspace{-0.5ex}
\end{displaymath}
with $R_s$ the source resistance, and $G_s = 1/R_s$. Besides
$G_{eq} = 1/R_{eq}$.
The expressions for $R_{eq}$ are given as:\par
\begin{center}
\begin{tabular}{||l|c||}\hline\hline
Amplifier & $Req$ \\\hline\hline
{\bfseries \sffamily Voltage} & $R_1||R_2$ \\\hline
{\bfseries \sffamily Transconductance} & $R$ \\\hline
{\bfseries \sffamily Transimpedance} & $R$ \\\hline
{\bfseries \sffamily Current} & $R_1 + R_2$ \\\hline\hline
\end{tabular}
\end{center}
It clearly results that this contribution depends on the value of the
source resistance and the values of the linear resistors implementing
the feedback network.
\item The noise from the passive network.\par
This is a noise contribution from the resistors that constitute the
gain of the amplifier. For the voltage and current amplifiers, the gain
involves two resistors ($R_1$ and $R_2$), and thus two terms in the
sum arise. For the other amplifiers, the gain is determined by a
single resistor, therefore only one term appears in the sum.
\end{itemize}

\begin{comment}
\subsection{Example}
The transconductance amplifier is used to show how to perform the
transformations above.

\begin{itemize}
\item {\bf Step 1} $u_{nn}$ and $u_{ns}$ of the Figure \ref{trans1}
are added, the result is $u_{n,eq,1}$ as shown in Figure \ref{trans2}.
\end{itemize}

\begin{figure}[hbtp]
\centerline{
\epsfxsize=50mm
\epsffile{/home/jarocho/tex/noise_castour/figuras/transadmittance_noise_amp.eps}}
\caption{Transadmittance Amplifier with Noise Sources Added.}
\label{trans1}
\end{figure} 

\begin{figure}[hbtp]
\centerline{
\epsfxsize=50mm
\epsffile{/home/jarocho/tex/noise_castour/figuras/transadmittance_1.eps}}
\caption{Voltage Noise Sources Added and I-Shift Applied.}
\label{trans2}
\end{figure} 

\begin{itemize}
\item {\bf Step 2} The I-shift is used on $i_{nn}$ as shown in Figure
\ref{trans2}. The Norton-Thevenin transformation is applied on $i_{nn}$
and $R_s$, the result is added to $u_{n,eq,1}$, this equalls to
$u_{n,eq,2}$ as shown in Figure \ref{trans3}.
The Figure \ref{trans2} shows $i_{nn}$ and $i_{nR}$ in parallel with
$R$, these noise sources are added and the result is $i_{n,eq,1}$ as
depicted in Figure \ref{trans3}.
\end{itemize}

\begin{figure}[hbtp]
\centerline{
\epsfxsize=50mm
\epsffile{/home/jarocho/tex/noise_castour/figuras/transadmittance_2.eps}}
\caption{Voltage Noise Sources Added and I-Shift Applied.}
\label{trans3}
\end{figure} 

\begin{itemize}
\item {\bf Step 3} The I-shift is applied on $i_{n,eq,1}$ as shown
in Figure \ref{trans3}. For the current noise source $i_{n,eq,1}$
located directly in parallel to the output port of the nullor, if we
apply the two-port transformation to the nullor it will show us that
this source will not yield any equivalent input noise source, since
all chain parameters of the nullor equal zero.
\end{itemize}

\begin{itemize}
\item {\bf Step 4} The two-port transformation is used on the source
$i_{n,eq,1}$ which is located at the amplier output. The result is
added with $u_{n,eq,2}$ and the result becomes the equivalent input
noise source $u_{n,eq,in}$ as shown in Figure \ref{trans4}.
Then a noiseless-amplifier is obtained.
\end{itemize}

\begin{figure}[hbtp]
\centerline{
\epsfxsize=50mm
\epsffile{/home/jarocho/tex/noise_castour/figuras/transadmittance_3.eps}}
\caption{Voltage Noise Sources Added and I-Shift Applied.}
\label{trans4}
\end{figure} 
\end{comment}

\section{Acknowledgments}
Roberto Casta\~neda Sheissa
is holder of a scholarship from CONACyT Mexico under contract
118652/120341.
This work has been partially supported by a CONACyT/Mexico
research project under grant 42588-Y.

\section{Conclusions}
The expressions of the equivalent input noise of the single-loop
configurations of nullor-based amplifiers have been obtained.
These are obtained by using simple transformations on the noise sources
present in the circuit.
The hierarchy of the expressions has been highlighted.
One important noise contribution comes from the source resistance.
The active network (the synthesised nullor) contributes with noise in both
forms, voltage and current. As expected, the noise of the active network
is closely related to the resistors of the configuration.
Finally, the feedback network adds noise to the expression.
The uniform structure of the equivalent input noise expressions
allows the designer to focuss the attention on the various sources of noise
present in the nullor-based configurations.

\bibliography{/home/jarocho/tex/bib/contin,/home/jarocho/tex/bib/nonlin,/home/jarocho/tex/bib/netwth,/home/jarocho/tex/bib/miscel,/home/jarocho/tex/bib/cad,/home/jarocho/tex/bib/bias,/home/jarocho/tex/bib/graph,/home/jarocho/tex/noise_castour/bib/noisecastour}
\end{document}

