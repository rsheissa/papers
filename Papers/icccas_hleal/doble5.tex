%\documentclass[twocolumn]{article}


\documentclass{article}
\usepackage{spconf}

\usepackage[dvips]{epsfig}
\usepackage{epsfig}
\usepackage{psfrag}
\usepackage{pifont}
\usepackage{amssymb}
\usepackage{amsmath}
\usepackage{rotate}
\usepackage{rotating}

\bibliographystyle{IEEEbib}

%
%	LINear ALGebra Macros
%
%  a) Matrix & vector alike manipulations
%
%

%%BUT firstly, a macro to set the first letter and word of
%%a paagraph in a bigger size than the rest of the text.
%% arg1=1stletter, arg2=rest_of_word
%% Example
%% /*****\HIS is a big T.
%%    *
%%    *
\def\PARAstart#1#2{\begingroup\def\par{\endgraf\endgroup\lineskiplimit=0pt}
    \setbox2=\hbox{\uppercase{#2} }\newdimen\tmpht \tmpht \ht2
    \advance\tmpht by \baselineskip\font\hhuge=cmr10 at \tmpht
    \setbox1=\hbox{{\hhuge #1}}
    \count7=\tmpht \count8=\ht1\divide\count8 by 1000 \divide\count7 by\count8
    \tmpht=.001\tmpht\multiply\tmpht by \count7\font\hhuge=cmr10 at \tmpht
    \setbox1=\hbox{{\hhuge #1}} \noindent \hangindent1.05\wd1
    \hangafter=-2 {\hskip-\hangindent \lower1\ht1\hbox{\raise1.0\ht2\copy1}%
    \kern-0\wd1}\copy2\lineskiplimit=-1000pt}


%
%	HORIZONTAL VECTORS	x1, x2, ... xn

%	x1, x2, .. xn
%
\newcommand{\hvec}
	   {\mbox{$x_1, x_2,\cdots,x_n$ }}

%	(x)1, (x)2, ... (x)n
%
\newcommand{\hvecx}[1]
	   {\mbox{${#1}_1, {#1}_2,\cdots,{#1}_n$ }}

%
%	(x)1, (y)2, ... (z)n
\newcommand{\hvecxyz}[3]
	   {\mbox{${#1}_1, {#2}_2,\cdots,{#3}_n$ }}

%
%	VERTICAL VECTORS

\newcommand{\vvec}
		{\mbox{$
			\begin{array}{c}
			x_1 \\ x_2 \\ \vdots \\ x_n \\
			\end{array}
		$}}

%
\newcommand{\vvecx}[1]
		{\mbox{$
			\begin{array}{c}
			#1_1 \\ #1_2 \\ \vdots \\ #1_n \\
			\end{array}
		$}}

%
\newcommand{\vvecxyz}[3]
		{\mbox{$
			\begin{array}{c}
			#1_1 \\ #2_2 \\ \vdots \\ #3_n \\
			\end{array}
		$}}

%
\newcommand{\dotcol}
{\mbox{$ 
	\begin{array}{c}
		\cdots\vspace{3mm}\\
		\cdots\vspace{3mm}\\
		\cdots\vspace{3mm}\\
		\cdots\vspace{3mm}\\
	\end{array}
$}}


%
%	GENERIC MATRICES


%
\newcommand{\genmtrx}[3]
{\mbox{$	%\left[
\pmatrix{
	#1_{11}	   & #1_{12}	& \cdots & #1_{1{#2}}	\cr 
	#1_{21}	   & #1_{22}	& \cdots & #1_{2{#2}}	\cr 
	\vdots 	   & \vdots 	& \ddots & \vdots	\cr
	#1_{{#3}1} & #1_{{#3}2} & \cdots & #1_{{#3}{#2}}\cr 
		%\rigth]
}
	$}}

%

%
				%% [ p1_{p2 p3} ]
\newcommand{\biggmtrx}[3]
{\mbox{$
	\left[ 
	\begin{array}{c}
		{#1}_{11}	     \vspace{3mm}\\ 
		{#1}_{21}	     \vspace{3mm}\\
		\vdots	     \vspace{3mm}\\ 
		{#1}_{{#3}1}   \vspace{3mm}\\ 
	\end{array}
	\begin{array}{c}
		{#1}_{12}      \vspace{3mm}\\ 
		{#1}_{22}      \vspace{3mm}\\
		\vdots       \vspace{3mm}\\ 
		{#1}_{{#3}2}   \vspace{3mm}\\ 
	\end{array}
	\begin{array}{c}
		\cdots\vspace{3mm}\\
		\cdots\vspace{3mm}\\
		\ddots\vspace{3mm}\\
		\cdots
	\end{array}
	\begin{array}{c}
		{#1}_{1{#2}}         \vspace{3mm}\\ 
		{#1}_{2{#2}}         \vspace{3mm}\\
		\vdots            \vspace{3mm}\\
		{#1}_{{#3}{#2}}     \vspace{3mm}\\
	\end{array}
	\right]
$}}

\input{/home/hleal/papers/doblelim/macros/math_aux.tex}
%
% List of macros defining
% Some Jacobians and matrices used for numerical
% items.
%


			% | d p1_p2 |
\newcommand{\dpart}[4]	%$|---------|$
			% | d p3_p4 |
{\mbox{$
	\frac{\partial {#1}_{#2} }
	     {\partial {#3}_{#4} }
      $}
}

\newcommand{\bigjac}[2]
{\mbox{$
	\left[ 
	\begin{array}{c}
		\dpart{#1}{1}{#2}{1} \vspace{3mm}\\ 
		\dpart{#1}{2}{#2}{1} \vspace{3mm}\\
		\vdots \vspace{3mm}\\ 
		\dpart{#1}{n}{#2}{1} \vspace{3mm}\\ 
	\end{array}
	\begin{array}{c}
		\dpart{#1}{1}{#2}{2} \vspace{3mm}\\ 
		\dpart{#1}{2}{#2}{2} \vspace{3mm}\\
		\vdots \vspace{3mm}\\
		\dpart{#1}{n}{#2}{2} \vspace{3mm}\\ 
	\end{array}
	\begin{array}{c}
		\cdots\vspace{3mm}\\
		\cdots\vspace{3mm}\\
		\ddots\vspace{3mm}\\
		\cdots
	\end{array}
	\begin{array}{c}
		\dpart{#1}{1}{#2}{n} \vspace{3mm}\\ 
		\dpart{#1}{2}{#2}{n} \vspace{3mm}\\
		\vdots \vspace{3mm} \\
		\dpart{#1}{n}{#2}{n} \vspace{3mm}\\
	\end{array}
	\right]
$}}


\newcommand{\chuajac}[3]
{\mbox{$
	\left[ 
	\begin{array}{c}
		\dpart{#1}{1}{#2}{1} \vspace{3mm}\\ 
		\dpart{#1}{2}{#2}{1} \vspace{3mm}\\
		\vdots \vspace{3mm}\\ 
		\dpart{#1}{n}{#2}{1} \vspace{3mm}\\ 
		0
	\end{array}
	\begin{array}{c}
		\dpart{#1}{1}{#2}{2} \vspace{3mm}\\ 
		\dpart{#1}{2}{#2}{2} \vspace{3mm}\\
		\vdots \vspace{3mm}\\
		\dpart{#1}{n}{#2}{2} \vspace{3mm}\\ 
		0
	\end{array}
	\begin{array}{c}
		\cdots\vspace{3mm}\\
		\cdots\vspace{3mm}\\
		\vdots\vspace{3mm}\\
		\cdots\vspace{3mm}\\
		\cdots
	\end{array}
	\begin{array}{c}
		\dpart{#1}{1}{#2}{k} \vspace{3mm}\\ 
		\dpart{#1}{2}{#2}{k} \vspace{3mm}\\
		\vdots \vspace{3mm} \\
		\dpart{#1}{n}{#2}{k} \vspace{3mm}\\
		sgn(\frac{dx_k}{dp})
	\end{array}
	\begin{array}{c}
		\cdots\vspace{3mm}\\
		\cdots\vspace{3mm}\\
		\ddots\vspace{3mm}\\
		\cdots\vspace{3mm}\\
		\cdots
	\end{array}
	\begin{array}{c}
		\dpart{#1}{1}{#2}{n} \vspace{3mm}\\ 
		\dpart{#1}{2}{#2}{n} \vspace{3mm}\\
		\vdots \vspace{3mm} \\
		\dpart{#1}{n}{#2}{n} \vspace{3mm}\\
		0
	\end{array}
	\begin{array}{c}
		\dpart{#1}{1}{#3}{} \vspace{3mm}\\ 
		\dpart{#1}{2}{#3}{} \vspace{3mm}\\
		\vdots \vspace{3mm}\\
		\dpart{#1}{n}{#3}{} \vspace{3mm}\\
		0
	\end{array}
	\right]
$}}
%==============
%==============

\newcommand{\jacob}[2]
			{\mbox{$ \left[
			{\mbox{$ 
				{\coljac}{#1}{#2}{1}
			$}}
			{\mbox{$ 
				{\coljac}{#1}{#2}{2}
			$}}
			{\mbox{$ 
				{\dotcol}
			$}}
			{\mbox{$ 
				{\coljac}{#1}{#2}{n}
			$}}

			\rigth] $}}



\newcommand{\jac}[2]
{\mbox{$	%\left[
\pmatrix{
{\frac{\partial #1_1}{\partial #2_1} }& 
	{\frac{\partial #1_1}{\partial #2_2} }& 
		\cdots &
			{\frac{\partial #1_1}{\partial #2_n} }\cr 
{\frac{\partial #1_2}{\partial #2_1} }& 
	{\frac{\partial #1_2}{\partial #2_2} }& 
		\cdots &
			{\frac{\partial #1_2}{\partial #2_n} }\cr 
\vdots& \vdots& \ddots& \vdots\cr
{\frac{\partial #1_n}{\partial #2_1} }& 
	{\frac{\partial #1_n}{\partial #2_2} }& 
		\cdots &
			{\frac{\partial #1_n}{\partial #2_n} }\cr
		%\rigth]
}
	$}}


\newcommand{\coljac}[3]
{\mbox{$ 
	\begin{array}{c}
		\frac{\partial #1_1}{\partial #2_#3}\vspace{3mm} \\ 
		\frac{\partial #1_2}{\partial #2_#3}\vspace{3mm} \\ 
		\vdots \vspace{3mm} \\
		\frac{\partial #1_n}{\partial #2_#3}\vspace{3mm} \\ 
	\end{array}
$}}

%
%	C and O subset alike macros
% \cmd = normal comand
% \mcmd = comand surrounded by $ ... $
%

%
% C(T, B)
\newcommand{\cset}{\cal{C}(\pig{T}, \pig{B})}
\newcommand{\oset}{\cal{O}(\pig{T}, \pig{B})}
\newcommand{\mcset}{\mbox{$\cal{C}(\pig{T}, \pig{B})$} }
\newcommand{\moset}{\mbox{$\cal{O}(\pig{T}, \pig{B})$} }

\newcommand{\obe}{{\cal{O}}^{be}}
\newcommand{\mobe}{\mbox{${\cal{O}}^{be}$ }}
\newcommand{\omix}{{\cal{O}}^{mix}}
\newcommand{\momix}{\mbox{${\cal{O}}^{mix}$ }}
\newcommand{\obc}{{\cal{O}}^{bc}}
\newcommand{\mobc}{\mbox{${\cal{O}}^{bc}$ }}

%\newcommand{\pig}[1]{\mbox{\boldmath ${#1}$}	}



\newtheorem{Theod}{{\bf Definition}}
\newtheorem{Theos}{{\bf Statement}}
\newtheorem{Theor}{{\bf Theorem}}
\newtheorem{Theoc}{{\bf Corallary}}

\setlength{\oddsidemargin}{-7mm}
\setlength{\evensidemargin}{-7mm}
\setlength{\topmargin}{-1mm}
\setlength{\textwidth}{176mm}
\setlength{\columnsep}{8mm}
\setlength{\textheight}{22.8cm}


\name{H\'ector V\'azquez-Leal, Luis Hern\'andez-Mart\'{\i}nez and Arturo Sarmiento-Reyes}

\address{National Institute for Astrophysics, Optics and Electronics \\
Electronics Department, CAD Group, P.O. Box 51, 72000, Puebla, Pue., Mexico \\
Email: {\tt hleal@inaoep.mx} 
}


\title{\vspace{1mm} Numerical Continuation Scheme for tracing the Double Bounded Homotopy for analysing  nonlinear circuits}


\begin{document}
\maketitle

\begin{abstract}
%REDACTAR TODO
A numerical continuation for tracing a novel homotopy method for obtaining  DC solutions of 
nonlinear circuits is proposed.
The homotopy method is called {\it double bounded homotopy} and it is used  to f\text{i}nd multiple DC solutions 
with the advantage of having
a stop criterion  which is based on the property of having a double bounded trajectory. 
The main properties of this homotopy are explained by using the Lambert $\cal W$ function.
F\text{i}nally, key aspects of the implementetion of the numerical continuation
for thes new homotopy are explained.
\end{abstract}


\section{Introduction}


The problem of f\text{i}nding the DC operating point is important because
it constitutes the starting point for other kinds of analyses such as
the ac small signal and transient analyses \cite{homo_ogrodzki}.
It consists in f\text{i}nding the roots of the nonlinear algebraic equations
(NAEs) emanating from nonlinear resistive networks, NAEs that have the
general form given by:
\begin{equation}
\pig{f}(\pig{x}) = \pig{0}
\label{fofx}
\end{equation}
where $\pig{x}$ are the unknowns accordingly to the circuit analysis method
used to set-up the equilibrium equation of the circuit.
\par
As it is well-known, equation (\ref{fofx}) may posses a single solution,
no solution at all, or multiple solutions.
In this paper, the attention is focused on the last case, where most
of the Newton-Raphson (NR) methods fail to converge. Homotopy methods
have been considered as a good choice of methods that overcome the
shortcomings of the NR-like methods
\cite{homo_ArtificialP},\cite{homo_DWolfMulti},\cite{cont_leu1}.


Homotopy methods are based on forming an auxiliary equation in order to
convert the problem of f\text{i}nding the roots of a NAE (a static problem)
into
a problem of f\text{i}nding the solution of an associated ordinary differential equation
(a dynamic problem)
\cite{homo_ogrodzki},\cite{cont_bra}. 
This auxiliary equation, also called the homotopy equation, is formed
by adding a parameter to the original equation:
\begin{equation}
\pig{H}(
\pig{f}(\pig{x}), \lambda
) = \pig{0}
\label{fofx2}
\end{equation}
where  $\pig{H}(\cdot)$ is the homotopy relationship, $\pig{f}(\pig{x})$ is the original equilibrium equation of
the circuit and $\lambda$ is the homotopy parameter.
The solution is found by using a {\it numerical path following} procedure, where
$\lambda$ is used as the integration parameter that varies from
an initial value
$\lambda_0$
to a value
$\lambda^{\ast}$ where a solution $\pig{x}^{\ast}$, to the original system
(\ref{fofx}) is found.
From this point, the integration procedure continues in order to search for
another solution.
The whole procedure results in tracing  a
path where the solutions lie.
%%Therefore, the homotopy trajectory is the set of solutions
%%of $\pig{H}^{-1}(\pig{0})$.

Although, homotopy methods are able to f\text{i}nd more than one solution to
(\ref{fofx}), they still exhibit several lacks.
Among them, it is worthy to mention:
\begin{itemize}
\item No global convergence. The capability of f\text{i}nding all solutions
cannot be always guaranteed. In fact, this depends on a series of issues,
such as the initial point where the homotopy starts to trace the solution,
the trace method and the type of nonlinear equation.
\item Stop criterion. Because there are two types of paths of solutions,
open and closed paths, the main problem is when to stop searching for
more solutions?. For closed paths, this can be solved by testing whether
a new solution has previously been found. For open paths, this is a serious
drawback, because there is no reasonable and reliable stop criterion
to decide when to stop seeking for more solutions.
\end{itemize}



%%%%%%%%%%%%%%%%%%%%%%%%%%%%%%%%%%%%%%%%%%%%%%%%%%%%%%%%%%%%%%%%%%%%%%%%%%%%%%%%%%%%%%%%%%%%%%%%%%%%%%%%%%%%%%%%%%%

In \cite{homo_green}, a method was proposed  to bound the homotopy trajectory (in order to have a stop criterion) whithin the interval $\lambda \in [0,2]$.
This method is applied under two condutions: f\text{i}rst, it is necessary to know {\it a priori} one DC solution in order to be used
as initial point in the method, and second, the circuit needs to be passive in the rank $\lambda \in [0,2]$.
The main goal of our work is to introduce a method for
creating a stop criterion for a homotopy that can be applied
to general nonlinear circuits.

\section{The doble bounded Homotopy}


The proposed homotopy is called {\bf double bounded homotopy} which is def\text{i}ned as
\begin{Theod}
The {\bf double bounded homotopy} is def\text{i}ned by
\begin{equation}
H(f(x),\lambda)=CQ+e^Q\ln(Df^2(x)+1)
\label{fx}
\end{equation}
where $f(x)$ is the function to solve, $\lambda$ is the homotopy parameter, $a$ and $b$ are the values where the homotopy funcion solutions
are the same than the solutions of $f(x)$,
$C$ and $D$ are positive constants of the homotopy, and $Q$ is given by:
\begin{displaymath}
(\lambda-a)(\lambda-b)
\end{displaymath}
\label{InestCond3}
\end{Theod}
The term $(Df^2(x)+1)$ must be used in order
to assure that the argument of the log-function  never becomes negative. This homotopy has the main propertie of
bounding the trajectory in both $x$- and $\lambda$ directions.


The double bounded homotopy has a  symmetry axis, which  gives  an important advantage that traditional homotopies
does not have, because it can
be used to shorten the lenght of the trajectory  by tracing only one of the symmetrical branches.
 The F\text{i}gure \ref{halftrack} depicts
the tracing strategy by shading the half side to be traced,
where the initial and f\text{i}nal points ($x_i$ and $x_f$ respectively) of the homotopy trajectory  are located
at the value of:

\begin{displaymath}
\begin{array}{l}
(x_i,\lambda_i)=(x_i,(a+b)/2)\\
(x_f,\lambda_f)=(x_f,(a+b)/2)
\end{array}
\end{displaymath}


\begin{figure}[hbtp]
\psfrag{xi}{{\small $[x_i,(a+b)/2]$}}
\psfrag{xf}{{\small $[x_f,(a+b)/2]$}}
\psfrag{o}{$\lambda$}
\centerline{
\epsfxsize=40mm
\epsffile{/home/hleal/papers/doblelim/figs/halftrack.eps}}
\caption{Numerical path following  based on symetrical properties}
\label{halftrack}
\end{figure}

\subsection{Properties}

The homotopy function  can be solved for $\lambda$. It has two solutions:
{
\small
\begin{displaymath}
\begin{array}{c}
\lambda_1=0.5(a+b)+0.5\sqrt{(b-a)^2-4{\cal{W}}(\ln(Df^2(x)+1)/C)} \\
\lambda_2=0.5(a+b)-0.5\sqrt{(b-a)^2-4{\cal{W}}(\ln(Df^2(x)+1)/C)}
\end{array}
\end{displaymath}
}
In fact,  $\lambda_1$ and $\lambda_2$ represent the mirrored branches of
the  homotopy trajectory. Analysis of these expressions allows us to
determine the ranks of values for the functions  $\lambda_1$ and $\lambda_2$:
\begin{displaymath}
\begin{array}{l}
0.5(a+b) \leqslant \lambda_1 \leqslant b\\\\
\hspace{13mm}a \leqslant \lambda_2 \leqslant 0.5(a+b)
\end{array}
\end{displaymath}
as shown in F\text{i}gure \ref{halftrack}.\par


The ranks of the branches in the  $x$-direction  are def\text{i}ned by the function $f(x)$, when  $f(x)$ satisf\text{i}es:

\begin{displaymath}
\begin{array}{l}
|f(x)|\leqslant(1/D)\sqrt{D(e^{(C/4)(b-a)^2 e^{(a-b)^2/4}}-1)}\\
\end{array}
\end{displaymath}



\section{Numerical Path following aspects}

The path following is a mixture of a variety of numerical methods, focussed on designing 
a path in order to accomplish specif\text{i}c needs of a homotopy formulation.\par

For the proposed homotopy formulation, the path
following consists in tracing the curve implicitly def\text{i}ned by Equation \ref{fx}.
The path following procedure resorts to a well-known predictor-corrector scheme.
The predictor  calculates the tangent to
$(x^j,\lambda_j)$ and using a step length calculates the $(\bar{x}^{j+1},\bar{\lambda}_{j+1})$
over the tangent. The corrector uses the solution of the predictor in order to obtain
a new point on the homotopy trajectory, given by  $(x^{j+1},\lambda_{j+1})$.
An important task whithin the predictor-corrector scheme is the step control.\par
Another important block is the so-called end-zero strategy that is 
triggered (by each solution) when the homotopy trajectory bounces in $\lambda=a$ or $\lambda=b$
depending on the branch where the solution lies.
F\text{i}nally, the path following stops tracing when the trajectory returns to the symmetry axis
$\lambda=(a+b)/2$.
F\text{i}gure \ref{bloques} depicts the basic blocks of the numerical path following procedure, where the dashed blocks
are the specif\text{i}c characteristics added to the procedure in order to include a stop criterion. 




\begin{figure}[h]
\psfrag{p}{{$(\bar{x}^{j+1},\bar{\lambda}_{j+1})$}}
\psfrag{c}{{$(x^{j+1},\lambda_{j+1})$}}
\psfrag{s}{{$h_{j+2}=h_{j+1}\zeta$}}
\psfrag{l1}{{\small $sign(\triangle \lambda_{j+1}  \triangle \lambda_j) \ne -1 $}}
\psfrag{l2}{{\small $ \lambda \ne (a+b)/2$}}
\psfrag{l}{{\small $\lambda=(a+b)/2$}}
\centerline{
\epsfxsize=44mm
\epsffile{/home/hleal/papers/iberchip/figs/bloques.eps}}
\caption{Block diagram of the numerical path following}
\label{bloques}
\end{figure}



\subsubsection{Predictor}

The predictor point for $(x^j,\lambda_j)$ is given by:

\begin{displaymath}
(\bar{\pig{x}}^{j+1},\bar{\lambda}_{j+1})=(\pig{x}^j,\lambda_j)+h*\pig{t}
\end{displaymath}
where $h$ is an appropriate step length and $\pig{t}$ is a normalised vector tangent to the
homotopy trajectory (see F\text{i}gure \ref{pcorrector}). This predictor can be considered as a step of the Euler
method for solving the differential equation that describes the homotopy trajectory (continuation path). 
Predictor steps are usually based on tangent predictions but there are
different alternatives like the secant predictor \cite{homo_seydel}, interpolation predictor
\cite{homo_allgower}, Taylor polynomial predictor \cite{homo_allgower}. 


\begin{figure}[hbtp]
\psfrag{A}{{$(x^j,\lambda_j)$}}
\psfrag{B}{{$(\bar{x}^{j+1},\bar{\lambda}_{j+1})$}}
\psfrag{C}{{$(x^{j+1},\lambda_{j+1})$}}
\centerline{
\epsfxsize=50mm
\epsffile{/home/hleal/papers/doblelim/figs/pcorrector.eps}}
\caption{Predictor-corrector steps}
\label{pcorrector}
\end{figure}

 
\subsubsection{Corrector}

When the predictor step f\text{i}nishes, it is necessary to f\text{i}t the homotopy trajectory by
using a corrector step. This step  solves the Equation \ref{fx} by starting
from $(\bar{x}^{j+1},\bar{\lambda}_{j+1})$ (see F\text{i}gure \ref{pcorrector}). The common
way to solve this equation is by using the Newton-Gauss method \cite{homo_allgower}, which
can solved for systems of the type $R^{N+1} \Rightarrow R^N$.

\subsubsection{Step Control}

The Predictor-Corrector scheme can be optimized for tracing acceleration by
using a step length control. A small constant step length can trace the curve successfully
but not eff\text{i}ciently, because this process involves too many steps along ``f\text{l}at'' branches.
Therefore, it is necessary to adapt the step length to the convergence behavior at each
predictor-corrector step. The basic criterion is to control the step by observing the
convergence quality of the corrector step. A change of the number of iterations in the corrector step produces
a compensation factor $\zeta$, which affects the step length as follows:


\begin{displaymath}
h_{j+2}=h_{j+1} \zeta
\label{zero}
\end{displaymath}


\subsubsection{F\text{i}nd Zero Strategy}

Without an eff\text{i}cient f\text{i}nding zero strategy
the path following is incomplete and the homotopy could fail to converge some operating
points (maybe not any solution at all).
The tracing of the homotopy trajectory begins in $\lambda=0$ and it ``ends''\footnote{In fact if it is wanted to f\text{i}nd multiple solutions the path
following should follow beyond $\lambda=1$.} when $\rho=1$. Therefore,
when the tracing is near to $\lambda=1$ the {\it f\text{i}nd Zero Strategy} begins. The most simple example
of strategy is to use $[\pig{x}_f,\lambda_f]$ (of the last iteration) like the initial point
to solve the $\pig{f}(\pig{x})$ with Newton method. 


Nevertheless, as Newton method has local convergence it still could fail
to f\text{i}nd the solution. In \cite{homo_sosonkina} are reported some techniques to do the {\it f\text{i}nd Zero Strategy} accurately and
reliable.  The basic idea is
to use two points ($p_1 < 1$ and $p_2 \geq 1$) around to $\lambda=1$, next, interpolate the point in $\lambda=1$ in order to
obtain a point nearly to the real solution and use a like Newton method to f\text{i}nd the solution
of the original system $\pig{f}(\pig{x})$.
The double bounded homotopy has the characteristic of never cross the $\lambda=1$, hence, the f\text{i}ndig zero strategy
should start when  bouncing in the bounding lines. A good way to achieve this process is to monitor the
change of sign of the  $\Delta\lambda$ produced in the predictor step. This can be done by multipplaying
the  $\Delta\lambda$ of two consecutive predictor steps.

\begin{displaymath}
sign(\triangle \lambda_{j+1}  \triangle \lambda_j) \ne -1 
\label{zero1}
\end{displaymath}



\section{Circuit Case}
F\text{i}gure \ref{chua} depict the chua's benchmark circuit of nine solutions. This circuit has 4 bipolar transistors modeled by the half-side Ebers-Moll model.

The equations system formulation is the same of \cite{homo_chua} which is based
on the brach voltages $v_1$, $v_2$, $v_3$, $v_4$. F\text{i}gure \ref{chuaf} depict
the homotopy trajectory and the six solutions found versus the branch voltage $v_1$.

\begin{figure}[h]
\centerline{
\epsfxsize=80mm
\epsffile{/home/hleal/papers/doblelim/figs/chiaf.eps}}
\caption{Solution of the Chua's circuit}
\label{chuaf}
\end{figure}


\begin{figure}[h]
\psfrag{F}{{\tiny $I_D=10^{-9}(e^{40v_D}-1)$}}
\psfrag{b}{{\tiny $\alpha=0.98$}}
\centerline{
\epsfxsize=80mm
\epsffile{/home/hleal/papers/iscas2004/figs/chua.eps}}
\caption{Chua's circuit with nine solutions}
\label{chua}
\end{figure}
%\newpage

\section{Conclusions}
In this work presents a numerical path following for tracing a new homotopy
method to obtain the operating points of nonlinear circuits having
multiple solutions. 
This homotopy method is called the double bounded homotopy and it overcomes the shortcomings of
traditional homotopy schemes regarding the stop criterion. The geometrical properties of the method have been
analysed by using the Lambert $\cal W$ function. The key aspects of the implementation of the numerical path following were depicted.



 

\bibliography{/home/hleal/papers/doblelim/bib/homoto,/home/hleal/papers/doblelim/bib/contin,/home/hleal/papers/doblelim/bib/netwth}
\end{document}









