
%% bare_conf.tex 
%% V1.2
%% 2002/11/18
%% by Michael Shell
%% mshell@ece.gatech.edu
%% 
%% NOTE: This text file uses MS Windows line feed conventions. When (human)
%% reading this file on other platforms, you may have to use a text
%% editor that can handle lines terminated by the MS Windows line feed
%% characters (0x0D 0x0A).
%% 
%% This is a skeleton file demonstrating the use of IEEEtran.cls 
%% (requires IEEEtran.cls version 1.6b or later) with an IEEE conference paper.
%% 
%% Support sites:
%% http://www.ieee.org
%% and/or
%% http://www.ctan.org/tex-archive/macros/latex/contrib/supported/IEEEtran/ 
%%
%% This code is offered as-is - no warranty - user assumes all risk.
%% Free to use, distribute and modify.

% *** Authors should verify (and, if needed, correct) their LaTeX system  ***
% *** with the testflow diagnostic prior to trusting their LaTeX platform ***
% *** with production work. IEEE's font choices can trigger bugs that do  ***
% *** not appear when using other class files.                            ***
% Testflow can be obtained at:
% http://www.ctan.org/tex-archive/macros/latex/contrib/supported/IEEEtran/testflow


% Note that the a4paper option is mainly intended so that authors in
% countries using A4 can easily print to A4 and see how their papers will
% look in print. Authors are encouraged to use U.S. letter paper when 
% submitting to IEEE. Use the testflow package mentioned above to verify
% correct handling of both paper sizes by the author's LaTeX system.
%
% Also note that the "draftcls" or "draftclsnofoot", not "draft", option
% should be used if it is desired that the figures are to be displayed in
% draft mode.
%
% This paper can be formatted using the peerreviewca
% (instead of conference) mode.
\documentclass[conference]{IEEEtran}
% If the IEEEtran.cls has not been installed into the LaTeX system files, 
% manually specify the path to it:
% \documentclass[conference]{../sty/IEEEtran} 


% some very useful LaTeX packages include:

%\usepackage{cite}      % Written by Donald Arseneau
                        % V1.6 and later of IEEEtran pre-defines the format
                        % of the cite.sty package \cite{} output to follow
                        % that of IEEE. Loading the cite package will
                        % result in citation numbers being automatically
                        % sorted and properly "ranged". i.e.,
                        % [1], [9], [2], [7], [5], [6]
                        % (without using cite.sty)
                        % will become:
                        % [1], [2], [5]--[7], [9] (using cite.sty)
                        % cite.sty's \cite will automatically add leading
                        % space, if needed. Use cite.sty's noadjust option
                        % (cite.sty V3.8 and later) if you want to turn this
                        % off. cite.sty is already installed on most LaTeX
                        % systems. The latest version can be obtained at:
                        % http://www.ctan.org/tex-archive/macros/latex/contrib/supported/cite/

%\usepackage{graphicx}  % Written by David Carlisle and Sebastian Rahtz
                        % Required if you want graphics, photos, etc.
                        % graphicx.sty is already installed on most LaTeX
                        % systems. The latest version and documentation can
                        % be obtained at:
                        % http://www.ctan.org/tex-archive/macros/latex/required/graphics/
                        % Another good source of documentation is "Using
                        % Imported Graphics in LaTeX2e" by Keith Reckdahl
                        % which can be found as esplatex.ps and epslatex.pdf
                        % at: http://www.ctan.org/tex-archive/info/
% NOTE: for dual use with latex and pdflatex, instead load graphicx like:
%\ifx\pdfoutput\undefined
%\usepackage{graphicx}
%\else
%\usepackage[pdftex]{graphicx}
%\fi

% However, be warned that pdflatex will require graphics to be in PDF
% (not EPS) format and will preclude the use of PostScript based LaTeX
% packages such as psfrag.sty and pstricks.sty. IEEE conferences typically
% allow PDF graphics (and hence pdfLaTeX). However, IEEE journals do not
% (yet) allow image formats other than EPS or TIFF. Therefore, authors of
% journal papers should use traditional LaTeX with EPS graphics.
%
% The path(s) to the graphics files can also be declared: e.g.,
% \graphicspath{{../eps/}{../ps/}}
% if the graphics files are not located in the same directory as the
% .tex file. This can be done in each branch of the conditional above
% (after graphicx is loaded) to handle the EPS and PDF cases separately.
% In this way, full path information will not have to be specified in
% each \includegraphics command.
%
% Note that, when switching from latex to pdflatex and vice-versa, the new
% compiler will have to be run twice to clear some warnings.


%\usepackage{psfrag}    % Written by Craig Barratt, Michael C. Grant,
                        % and David Carlisle
                        % This package allows you to substitute LaTeX
                        % commands for text in imported EPS graphic files.
                        % In this way, LaTeX symbols can be placed into
                        % graphics that have been generated by other
                        % applications. You must use latex->dvips->ps2pdf
                        % workflow (not direct pdf output from pdflatex) if
                        % you wish to use this capability because it works
                        % via some PostScript tricks. Alternatively, the
                        % graphics could be processed as separate files via
                        % psfrag and dvips, then converted to PDF for
                        % inclusion in the main file which uses pdflatex.
                        % Docs are in "The PSfrag System" by Michael C. Grant
                        % and David Carlisle. There is also some information 
                        % about using psfrag in "Using Imported Graphics in
                        % LaTeX2e" by Keith Reckdahl which documents the
                        % graphicx package (see above). The psfrag package
                        % and documentation can be obtained at:
                        % http://www.ctan.org/tex-archive/macros/latex/contrib/supported/psfrag/

%\usepackage{subfigure} % Written by Steven Douglas Cochran
                        % This package makes it easy to put subfigures
                        % in your figures. i.e., "figure 1a and 1b"
                        % Docs are in "Using Imported Graphics in LaTeX2e"
                        % by Keith Reckdahl which also documents the graphicx
                        % package (see above). subfigure.sty is already
                        % installed on most LaTeX systems. The latest version
                        % and documentation can be obtained at:
                        % http://www.ctan.org/tex-archive/macros/latex/contrib/supported/subfigure/

%\usepackage{url}       % Written by Donald Arseneau
                        % Provides better support for handling and breaking
                        % URLs. url.sty is already installed on most LaTeX
                        % systems. The latest version can be obtained at:
                        % http://www.ctan.org/tex-archive/macros/latex/contrib/other/misc/
                        % Read the url.sty source comments for usage information.

%\usepackage{stfloats}  % Written by Sigitas Tolusis
                        % Gives LaTeX2e the ability to do double column
                        % floats at the bottom of the page as well as the top.
                        % (e.g., "\begin{figure*}[!b]" is not normally
                        % possible in LaTeX2e). This is an invasive package
                        % which rewrites many portions of the LaTeX2e output
                        % routines. It may not work with other packages that
                        % modify the LaTeX2e output routine and/or with other
                        % versions of LaTeX. The latest version and
                        % documentation can be obtained at:
                        % http://www.ctan.org/tex-archive/macros/latex/contrib/supported/sttools/
                        % Documentation is contained in the stfloats.sty
                        % comments as well as in the presfull.pdf file.
                        % Do not use the stfloats baselinefloat ability as
                        % IEEE does not allow \baselineskip to stretch.
                        % Authors submitting work to the IEEE should note
                        % that IEEE rarely uses double column equations and
                        % that authors should try to avoid such use.
                        % Do not be tempted to use the cuted.sty or
                        % midfloat.sty package (by the same author) as IEEE
                        % does not format its papers in such ways.

%\usepackage{amsmath}   % From the American Mathematical Society
                        % A popular package that provides many helpful commands
                        % for dealing with mathematics. Note that the AMSmath
                        % package sets \interdisplaylinepenalty to 10000 thus
                        % preventing page breaks from occurring within multiline
                        % equations. Use:
%\interdisplaylinepenalty=2500
                        % after loading amsmath to restore such page breaks
                        % as IEEEtran.cls normally does. amsmath.sty is already
                        % installed on most LaTeX systems. The latest version
                        % and documentation can be obtained at:
                        % http://www.ctan.org/tex-archive/macros/latex/required/amslatex/math/



% Other popular packages for formatting tables and equations include:

%\usepackage{array}
% Frank Mittelbach's and David Carlisle's array.sty which improves the
% LaTeX2e array and tabular environments to provide better appearances and
% additional user controls. array.sty is already installed on most systems.
% The latest version and documentation can be obtained at:
% http://www.ctan.org/tex-archive/macros/latex/required/tools/

% Mark Wooding's extremely powerful MDW tools, especially mdwmath.sty and
% mdwtab.sty which are used to format equations and tables, respectively.
% The MDWtools set is already installed on most LaTeX systems. The lastest
% version and documentation is available at:
% http://www.ctan.org/tex-archive/macros/latex/contrib/supported/mdwtools/


% V1.6 of IEEEtran contains the IEEEeqnarray family of commands that can
% be used to generate multiline equations as well as matrices, tables, etc.


% Also of notable interest:

% Scott Pakin's eqparbox package for creating (automatically sized) equal
% width boxes. Available:
% http://www.ctan.org/tex-archive/macros/latex/contrib/supported/eqparbox/



% Notes on hyperref:
% IEEEtran.cls attempts to be compliant with the hyperref package, written
% by Heiko Oberdiek and Sebastian Rahtz, which provides hyperlinks within
% a document as well as an index for PDF files (produced via pdflatex).
% However, it is a tad difficult to properly interface LaTeX classes and
% packages with this (necessarily) complex and invasive package. It is
% recommended that hyperref not be used for work that is to be submitted
% to the IEEE. Users who wish to use hyperref *must* ensure that their
% hyperref version is 6.72u or later *and* IEEEtran.cls is version 1.6b 
% or later. The latest version of hyperref can be obtained at:
%
% http://www.ctan.org/tex-archive/macros/latex/contrib/supported/hyperref/
%
% Also, be aware that cite.sty (as of version 3.9, 11/2001) and hyperref.sty
% (as of version 6.72t, 2002/07/25) do not work optimally together.
% To mediate the differences between these two packages, IEEEtran.cls, as
% of v1.6b, predefines a command that fools hyperref into thinking that
% the natbib package is being used - causing it not to modify the existing
% citation commands, and allowing cite.sty to operate as normal. However,
% as a result, citation numbers will not be hyperlinked. Another side effect
% of this approach is that the natbib.sty package will not properly load
% under IEEEtran.cls. However, current versions of natbib are not capable
% of compressing and sorting citation numbers in IEEE's style - so this
% should not be an issue. If, for some strange reason, the user wants to
% load natbib.sty under IEEEtran.cls, the following code must be placed
% before natbib.sty can be loaded:
%
% \makeatletter
% \let\NAT@parse\undefined
% \makeatother
%
% Hyperref should be loaded differently depending on whether pdflatex
% or traditional latex is being used:
%
%\ifx\pdfoutput\undefined
%\usepackage[hypertex]{hyperref}
%\else
%\usepackage[pdftex,hypertexnames=false]{hyperref}
%\fi
%
% Pdflatex produces superior hyperref results and is the recommended
% compiler for such use.



% *** Do not adjust lengths that control margins, column widths, etc. ***
% *** Do not use packages that alter fonts (such as pslatex).         ***
% There should be no need to do such things with IEEEtran.cls V1.6 and later.


% correct bad hyphenation here
\hyphenation{op-tical net-works semi-conduc-tor IEEEtran}




%
%	LINear ALGebra Macros
%
%  a) Matrix & vector alike manipulations
%
%

%%BUT firstly, a macro to set the first letter and word of
%%a paagraph in a bigger size than the rest of the text.
%% arg1=1stletter, arg2=rest_of_word
%% Example
%% /*****\HIS is a big T.
%%    *
%%    *
\def\PARAstart#1#2{\begingroup\def\par{\endgraf\endgroup\lineskiplimit=0pt}
    \setbox2=\hbox{\uppercase{#2} }\newdimen\tmpht \tmpht \ht2
    \advance\tmpht by \baselineskip\font\hhuge=cmr10 at \tmpht
    \setbox1=\hbox{{\hhuge #1}}
    \count7=\tmpht \count8=\ht1\divide\count8 by 1000 \divide\count7 by\count8
    \tmpht=.001\tmpht\multiply\tmpht by \count7\font\hhuge=cmr10 at \tmpht
    \setbox1=\hbox{{\hhuge #1}} \noindent \hangindent1.05\wd1
    \hangafter=-2 {\hskip-\hangindent \lower1\ht1\hbox{\raise1.0\ht2\copy1}%
    \kern-0\wd1}\copy2\lineskiplimit=-1000pt}


%
%	HORIZONTAL VECTORS	x1, x2, ... xn

%	x1, x2, .. xn
%
\newcommand{\hvec}
	   {\mbox{$x_1, x_2,\cdots,x_n$ }}

%	(x)1, (x)2, ... (x)n
%
\newcommand{\hvecx}[1]
	   {\mbox{${#1}_1, {#1}_2,\cdots,{#1}_n$ }}

%
%	(x)1, (y)2, ... (z)n
\newcommand{\hvecxyz}[3]
	   {\mbox{${#1}_1, {#2}_2,\cdots,{#3}_n$ }}

%
%	VERTICAL VECTORS

\newcommand{\vvec}
		{\mbox{$
			\begin{array}{c}
			x_1 \\ x_2 \\ \vdots \\ x_n \\
			\end{array}
		$}}

%
\newcommand{\vvecx}[1]
		{\mbox{$
			\begin{array}{c}
			#1_1 \\ #1_2 \\ \vdots \\ #1_n \\
			\end{array}
		$}}

%
\newcommand{\vvecxyz}[3]
		{\mbox{$
			\begin{array}{c}
			#1_1 \\ #2_2 \\ \vdots \\ #3_n \\
			\end{array}
		$}}

%
\newcommand{\dotcol}
{\mbox{$ 
	\begin{array}{c}
		\cdots\vspace{3mm}\\
		\cdots\vspace{3mm}\\
		\cdots\vspace{3mm}\\
		\cdots\vspace{3mm}\\
	\end{array}
$}}


%
%	GENERIC MATRICES


%
\newcommand{\genmtrx}[3]
{\mbox{$	%\left[
\pmatrix{
	#1_{11}	   & #1_{12}	& \cdots & #1_{1{#2}}	\cr 
	#1_{21}	   & #1_{22}	& \cdots & #1_{2{#2}}	\cr 
	\vdots 	   & \vdots 	& \ddots & \vdots	\cr
	#1_{{#3}1} & #1_{{#3}2} & \cdots & #1_{{#3}{#2}}\cr 
		%\rigth]
}
	$}}

%

%
				%% [ p1_{p2 p3} ]
\newcommand{\biggmtrx}[3]
{\mbox{$
	\left[ 
	\begin{array}{c}
		{#1}_{11}	     \vspace{3mm}\\ 
		{#1}_{21}	     \vspace{3mm}\\
		\vdots	     \vspace{3mm}\\ 
		{#1}_{{#3}1}   \vspace{3mm}\\ 
	\end{array}
	\begin{array}{c}
		{#1}_{12}      \vspace{3mm}\\ 
		{#1}_{22}      \vspace{3mm}\\
		\vdots       \vspace{3mm}\\ 
		{#1}_{{#3}2}   \vspace{3mm}\\ 
	\end{array}
	\begin{array}{c}
		\cdots\vspace{3mm}\\
		\cdots\vspace{3mm}\\
		\ddots\vspace{3mm}\\
		\cdots
	\end{array}
	\begin{array}{c}
		{#1}_{1{#2}}         \vspace{3mm}\\ 
		{#1}_{2{#2}}         \vspace{3mm}\\
		\vdots            \vspace{3mm}\\
		{#1}_{{#3}{#2}}     \vspace{3mm}\\
	\end{array}
	\right]
$}}


%
% List of macros defining
% Some Jacobians and matrices used for numerical
% items.
%


			% | d p1_p2 |
\newcommand{\dpart}[4]	%$|---------|$
			% | d p3_p4 |
{\mbox{$
	\frac{\partial {#1}_{#2} }
	     {\partial {#3}_{#4} }
      $}
}

\newcommand{\bigjac}[2]
{\mbox{$
	\left[ 
	\begin{array}{c}
		\dpart{#1}{1}{#2}{1} \vspace{3mm}\\ 
		\dpart{#1}{2}{#2}{1} \vspace{3mm}\\
		\vdots \vspace{3mm}\\ 
		\dpart{#1}{n}{#2}{1} \vspace{3mm}\\ 
	\end{array}
	\begin{array}{c}
		\dpart{#1}{1}{#2}{2} \vspace{3mm}\\ 
		\dpart{#1}{2}{#2}{2} \vspace{3mm}\\
		\vdots \vspace{3mm}\\
		\dpart{#1}{n}{#2}{2} \vspace{3mm}\\ 
	\end{array}
	\begin{array}{c}
		\cdots\vspace{3mm}\\
		\cdots\vspace{3mm}\\
		\ddots\vspace{3mm}\\
		\cdots
	\end{array}
	\begin{array}{c}
		\dpart{#1}{1}{#2}{n} \vspace{3mm}\\ 
		\dpart{#1}{2}{#2}{n} \vspace{3mm}\\
		\vdots \vspace{3mm} \\
		\dpart{#1}{n}{#2}{n} \vspace{3mm}\\
	\end{array}
	\right]
$}}


\newcommand{\chuajac}[3]
{\mbox{$
	\left[ 
	\begin{array}{c}
		\dpart{#1}{1}{#2}{1} \vspace{3mm}\\ 
		\dpart{#1}{2}{#2}{1} \vspace{3mm}\\
		\vdots \vspace{3mm}\\ 
		\dpart{#1}{n}{#2}{1} \vspace{3mm}\\ 
		0
	\end{array}
	\begin{array}{c}
		\dpart{#1}{1}{#2}{2} \vspace{3mm}\\ 
		\dpart{#1}{2}{#2}{2} \vspace{3mm}\\
		\vdots \vspace{3mm}\\
		\dpart{#1}{n}{#2}{2} \vspace{3mm}\\ 
		0
	\end{array}
	\begin{array}{c}
		\cdots\vspace{3mm}\\
		\cdots\vspace{3mm}\\
		\vdots\vspace{3mm}\\
		\cdots\vspace{3mm}\\
		\cdots
	\end{array}
	\begin{array}{c}
		\dpart{#1}{1}{#2}{k} \vspace{3mm}\\ 
		\dpart{#1}{2}{#2}{k} \vspace{3mm}\\
		\vdots \vspace{3mm} \\
		\dpart{#1}{n}{#2}{k} \vspace{3mm}\\
		sgn(\frac{dx_k}{dp})
	\end{array}
	\begin{array}{c}
		\cdots\vspace{3mm}\\
		\cdots\vspace{3mm}\\
		\ddots\vspace{3mm}\\
		\cdots\vspace{3mm}\\
		\cdots
	\end{array}
	\begin{array}{c}
		\dpart{#1}{1}{#2}{n} \vspace{3mm}\\ 
		\dpart{#1}{2}{#2}{n} \vspace{3mm}\\
		\vdots \vspace{3mm} \\
		\dpart{#1}{n}{#2}{n} \vspace{3mm}\\
		0
	\end{array}
	\begin{array}{c}
		\dpart{#1}{1}{#3}{} \vspace{3mm}\\ 
		\dpart{#1}{2}{#3}{} \vspace{3mm}\\
		\vdots \vspace{3mm}\\
		\dpart{#1}{n}{#3}{} \vspace{3mm}\\
		0
	\end{array}
	\right]
$}}
%==============
%==============

\newcommand{\jacob}[2]
			{\mbox{$ \left[
			{\mbox{$ 
				{\coljac}{#1}{#2}{1}
			$}}
			{\mbox{$ 
				{\coljac}{#1}{#2}{2}
			$}}
			{\mbox{$ 
				{\dotcol}
			$}}
			{\mbox{$ 
				{\coljac}{#1}{#2}{n}
			$}}

			\rigth] $}}



\newcommand{\jac}[2]
{\mbox{$	%\left[
\pmatrix{
{\frac{\partial #1_1}{\partial #2_1} }& 
	{\frac{\partial #1_1}{\partial #2_2} }& 
		\cdots &
			{\frac{\partial #1_1}{\partial #2_n} }\cr 
{\frac{\partial #1_2}{\partial #2_1} }& 
	{\frac{\partial #1_2}{\partial #2_2} }& 
		\cdots &
			{\frac{\partial #1_2}{\partial #2_n} }\cr 
\vdots& \vdots& \ddots& \vdots\cr
{\frac{\partial #1_n}{\partial #2_1} }& 
	{\frac{\partial #1_n}{\partial #2_2} }& 
		\cdots &
			{\frac{\partial #1_n}{\partial #2_n} }\cr
		%\rigth]
}
	$}}


\newcommand{\coljac}[3]
{\mbox{$ 
	\begin{array}{c}
		\frac{\partial #1_1}{\partial #2_#3}\vspace{3mm} \\ 
		\frac{\partial #1_2}{\partial #2_#3}\vspace{3mm} \\ 
		\vdots \vspace{3mm} \\
		\frac{\partial #1_n}{\partial #2_#3}\vspace{3mm} \\ 
	\end{array}
$}}

%
%	C and O subset alike macros
% \cmd = normal comand
% \mcmd = comand surrounded by $ ... $
%

%
% C(T, B)
\newcommand{\cset}{\cal{C}(\pig{T}, \pig{B})}
\newcommand{\oset}{\cal{O}(\pig{T}, \pig{B})}
\newcommand{\mcset}{\mbox{$\cal{C}(\pig{T}, \pig{B})$} }
\newcommand{\moset}{\mbox{$\cal{O}(\pig{T}, \pig{B})$} }

\newcommand{\obe}{{\cal{O}}^{be}}
\newcommand{\mobe}{\mbox{${\cal{O}}^{be}$ }}
\newcommand{\omix}{{\cal{O}}^{mix}}
\newcommand{\momix}{\mbox{${\cal{O}}^{mix}$ }}
\newcommand{\obc}{{\cal{O}}^{bc}}
\newcommand{\mobc}{\mbox{${\cal{O}}^{bc}$ }}

%\newcommand{\pig}[1]{\mbox{\boldmath ${#1}$}	}




%\usepackage{spconf}
\newtheorem{Theod}{{\bf Def\text{i}nition}}
\newtheorem{Theos}{{\bf Statement}}
\newtheorem{Theor}{{\bf Theorem}}
\newtheorem{Theoc}{{\bf Corallary}}

\usepackage[dvips]{epsfig}
\usepackage{epsfig}
\usepackage{psfrag}
\usepackage{pifont}
\usepackage{amssymb}
\usepackage{amsmath}
\usepackage{rotate}
\usepackage{rotating}




\begin{document}

% paper title
\title{Numerical continuation scheme for tracing the double bounded homotopy for analysing  nonlinear circuits}


% author names and affiliations
% use a multiple column layout for up to three different
% affiliations
\author{\authorblockN{H\'ector V\'azquez-Leal}
\authorblockA{National Institute for Astrophysics,\\
Optics and Electronics\\
Electronics Department, CAD Group, \\
P.O. Box 51, 72000, Puebla, Pue., Mexico \\
Email: hleal@inaoep.mx}
\and
\authorblockN{Luis Hern\'andez-Mart\'{\i}nez}
\authorblockA{Laboratory of Nonlinear Systems,\\ School of Computer Sciences,\\ Swiss Federal Institute of \\ Technology Lausanne,\\
Lausanne, Switzerland}
\and
\authorblockN{Arturo Sarmiento-Reyes} 
\authorblockA{National Institute for Astrophysics,\\
Optics and Electronics\\
Electronics Department, CAD Group, \\
P.O. Box 51, 72000, Puebla, Pue., Mexico}}


% avoiding spaces at the end of the author lines is not a problem with
% conference papers because we don't use \thanks or \IEEEmembership


% for over three affiliations, or if they all won't fit within the width
% of the page, use this alternative format:
% 
%\author{\authorblockN{Michael Shell\authorrefmark{1},
%Homer Simpson\authorrefmark{2},
%James Kirk\authorrefmark{3}, 
%Montgomery Scott\authorrefmark{3} and
%Eldon Tyrell\authorrefmark{4}}
%\authorblockA{\authorrefmark{1}School of Electrical and Computer Engineering\\
%Georgia Institute of Technology,
%Atlanta, Georgia 30332--0250\\ Email: mshell@ece.gatech.edu}
%\authorblockA{\authorrefmark{2}Twentieth Century Fox, Springfield, USA\\
%Email: homer@thesimpsons.com}
%\authorblockA{\authorrefmark{3}Starfleet Academy, San Francisco, California 96678-2391\\
%Telephone: (800) 555--1212, Fax: (888) 555--1212}
%\authorblockA{\authorrefmark{4}Tyrell Inc., 123 Replicant Street, Los Angeles, California 90210--4321}}



% use only for invited papers
%\specialpapernotice{(Invited Paper)}

% make the title area
\maketitle



\begin{abstract}
%REDACTAR TODO
A numerical continuation for tracing a novel homotopy method for obtaining  DC solutions of 
nonlinear circuits is proposed.
The homotopy method is called {\it double bounded homotopy} and it is used  to f\text{i}nd multiple DC solutions 
with the advantage of having
a stop criterion  which is based on the property of having a double bounded trajectory. 
The key aspects of the implementetion of the numerical continuation
for this double bounded homotopy are explained.
\end{abstract}


\section{Introduction}



The homotopy has an important role in the analysis of multiple operating points circuits.
Some articles, regard to homotopies are aplied to particular kind of circuits. However, it
is necessary to  




The problem of f\text{i}nding the DC operating point is important because
it constitutes the starting point for other kinds of analyses such as
the ac small signal and transient analyses \cite{homo_ogrodzki}.
It consists in f\text{i}nding the roots of the nonlinear algebraic equations
(NAEs) emanating from nonlinear resistive networks, NAEs that have the
general form given by:
\begin{equation}
\pig{f}(\pig{x}) = \pig{0}
\label{fofx}
\end{equation}
where $\pig{x}$ are the unknowns accordingly to the circuit analysis method
used to set-up the equilibrium equation of the circuit.
\par
As it is well-known, equation (\ref{fofx}) may posses a single solution,
no solution at all, or multiple solutions.
In this paper, the attention is focused on the last case, where most
of the Newton-Raphson (NR) methods fail to converge. Homotopy methods
have been considered as a good choice of methods that overcome the
shortcomings of the NR-like methods
\cite{homo_ArtificialP},\cite{homo_DWolfMulti},\cite{cont_leu1}.


Homotopy methods are based on forming an auxiliary equation in order to
convert the problem of f\text{i}nding the roots of a NAE (a static problem)
into
a problem of f\text{i}nding the solution of an associated ordinary differential equation
(a dynamic problem)
\cite{homo_ogrodzki},\cite{cont_bra}. 
This auxiliary equation, also called the homotopy equation, is formed
by adding a parameter to the original equation:
\begin{equation}
\pig{H}(
\pig{f}(\pig{x}), \lambda
) = \pig{0}
\label{fofx2}
\end{equation}
where  $\pig{H}(\cdot)$ is the homotopy relationship, $\pig{f}(\pig{x})$ is the original equilibrium equation of
the circuit and $\lambda$ is the homotopy parameter.
The solution is found by using a {\it numerical continuatio} procedure, where
$\lambda$ is used as the integration parameter that varies from
an initial value
$\lambda_0$
to a value
$\lambda^{\ast}$ where a solution $\pig{x}^{\ast}$, to the original system
(\ref{fofx}) is found.
From this point, the integration procedure continues in order to search for
another solution.
The whole procedure results in tracing  a
path where the solutions lie.
%%Therefore, the homotopy trajectory is the set of solutions
%%of $\pig{H}^{-1}(\pig{0})$.



Although, homotopy methods are able to f\text{i}nd more than one solution to
(\ref{fofx}), they still exhibit several lacks. Among them, it is worthy to mention the {\bf stop criterion}. 
There are two types of paths of solutions,
open and closed paths, the main problem is when to stop searching for
more solutions?. For closed paths, this can be solved by testing whether
a new solution has previously been found. For open paths, this is a serious
drawback, because there is no reasonable and reliable stop criterion
to decide when to stop seeking for more solutions.




%%%%%%%%%%%%%%%%%%%%%%%%%%%%%%%%%%%%%%%%%%%%%%%%%%%%%%%%%%%%%%%%%%%%%%%%%%%%%%%%%%%%%%%%%%%%%%%%%%%%%%%%%%%%%%%%%%%

In \cite{homo_green}, a method was proposed  to bound the homotopy trajectory (in order to have a stop criterion) whithin the interval $\lambda \in [0,2]$.
This method is applied under two condutions: f\text{i}rst, it is necessary to know {\it a priori} one DC solution in order to be used
as initial point in the method, and second, the circuit needs to be passive in the rank $\lambda \in [0,2]$.
The doble bounded Homotopy is proposed in order to circunvent the problem of stop criterion. Besides, a modif\text{i}ed
numerical continuation is introduced in order to trace the homotopy trajectories and  implement the stop criterion.


\section{The doble bounded Homotopy}


The proposed homotopy is called {\bf double bounded homotopy} which is def\text{i}ned as
\begin{Theod}
The {\bf double bounded homotopy} is def\text{i}ned by
\begin{equation}
H(f(x),\lambda)=CQ+e^Q\ln(Df^2(x)+1)
\label{fx}
\end{equation}
where $f(x)$ is the function to solve, $\lambda$ is the homotopy parameter, $a$ and $b$ are the values where the homotopy funcion solutions
are the same than the solutions of $f(x)$,
$C$ and $D$ are positive constants of the homotopy, and $Q$ is given by:
\begin{displaymath}
(\lambda-a)(\lambda-b)
\end{displaymath}
\label{InestCond3}
\end{Theod}
The term $(Df^2(x)+1)$ must be used in order
to assure that the argument of the log-function  never becomes negative. This homotopy has the main propertie of
bounding the trajectory in both $x$- and $\lambda$ directions.


The double bounded homotopy has a  symmetry axis, which  gives  an important advantage that traditional homotopies
does not have, because it can
be used to shorten the lenght of the trajectory  by tracing only one of the symmetrical branches.
 The F\text{i}gure \ref{halftrack} depicts
the tracing strategy by shading the half side to be traced,
where the initial and f\text{i}nal points ($x_i$ and $x_f$ respectively) of the homotopy trajectory  are located
at the value of:

\begin{displaymath}
\begin{array}{l}
(x_i,\lambda_i)=(x_i,(a+b)/2)\\
(x_f,\lambda_f)=(x_f,(a+b)/2)
\end{array}
\end{displaymath}


\begin{figure}[hbtp]
\psfrag{xi}{{\small $[x_i,(a+b)/2]$}}
\psfrag{xf}{{\small $[x_f,(a+b)/2]$}}
\psfrag{o}{$\lambda$}
\centerline{
\epsfxsize=80mm
\epsffile{/home/hleal/papers/doblelim/figs/halftrack.eps}}
\caption{Numerical continuation  based on symetrical properties}
\label{halftrack}
\end{figure}

\subsection{Properties}

The homotopy function  can be solved for $\lambda$. It has two solutions:
{
\small
\begin{displaymath}
\begin{array}{c}
\lambda_1=0.5(a+b)+0.5\sqrt{(b-a)^2-4{\cal{W}}(\ln(Df^2(x)+1)/C)} \\
\lambda_2=0.5(a+b)-0.5\sqrt{(b-a)^2-4{\cal{W}}(\ln(Df^2(x)+1)/C)}
\end{array}
\end{displaymath}
}
In fact,  $\lambda_1$ and $\lambda_2$ represent the mirrored branches of
the  homotopy trajectory. Analysis of these expressions allows us to
determine the ranks of values for the functions  $\lambda_1$ and $\lambda_2$:
\begin{displaymath}
\begin{array}{l}
0.5(a+b) \leqslant \lambda_1 \leqslant b\\\\
\hspace{13mm}a \leqslant \lambda_2 \leqslant 0.5(a+b)
\end{array}
\end{displaymath}
as shown in F\text{i}gure \ref{halftrack}.\par


The ranks of the branches in the  $x$-direction  are def\text{i}ned by the function $f(x)$, when  $f(x)$ satisf\text{i}es:

\begin{displaymath}
\begin{array}{l}
|f(x)|\leqslant(1/D)\sqrt{D(e^{(C/4)(b-a)^2 e^{(a-b)^2/4}}-1)}\\
\end{array}
\end{displaymath}



\section{Modif\text{i}ed numerical continuation}

The the numerical continuation is a mixture of a variety of numerical methods, focussed on designing 
a path in order to accomplish specif\text{i}c needs of a homotopy formulation.\par

For the proposed homotopy formulation, the numerical continuation consists in tracing the curve implicitly def\text{i}ned by Equation \ref{fx}.
The numerical continuation procedure resorts to a well-known predictor-corrector scheme.
The predictor  calculates the tangent to
$(x^j,\lambda_j)$ and using a step length calculates the $(\bar{x}^{j+1},\bar{\lambda}_{j+1})$
over the tangent. The corrector uses the solution of the predictor in order to obtain
a new point on the homotopy trajectory, given by  $(x^{j+1},\lambda_{j+1})$.
An important task whithin the predictor-corrector scheme is the step control.\par
Another important block is the so-called end-zero strategy that is 
triggered (by each solution) when the homotopy trajectory bounces in $\lambda=a$ or $\lambda=b$
depending on the branch where the solution lies.
F\text{i}nally, the numerical continuation stops tracing when the trajectory returns to the symmetry axis
$\lambda=(a+b)/2$.
F\text{i}gure \ref{bloques} depicts the basic blocks of the numerical continuation procedure, where the dashed blocks
are the specif\text{i}c characteristics added to the procedure in order to include a stop criterion. 




\begin{figure}[h]
\psfrag{p}{{$(\bar{x}^{j+1},\bar{\lambda}_{j+1})$}}
\psfrag{c}{{$(x^{j+1},\lambda_{j+1})$}}
\psfrag{s}{{$h_{j+2}=h_{j+1}\zeta$}}
\psfrag{l1}{{\small $sign(\triangle \lambda_{j+1}  \triangle \lambda_j) \ne -1 $}}
\psfrag{l2}{{\small $ \lambda \ne (a+b)/2$}}
\psfrag{l}{{\small $\lambda=(a+b)/2$}}
\centerline{
\epsfxsize=54mm
\epsffile{/home/hleal/papers/icccas/Template_LaTex/figs/bloques.eps}}
\caption{Block diagram of the modif\text{i}ed numerical continuation}
\label{bloques}
\end{figure}



\subsubsection{Predictor}

The predictor point for $(x^j,\lambda_j)$ is given by:

\begin{displaymath}
(\bar{\pig{x}}^{j+1},\bar{\lambda}_{j+1})=(\pig{x}^j,\lambda_j)+h*\pig{t}
\end{displaymath}
where $h$ is an appropriate step length and $\pig{t}$ is a normalised vector tangent to the
homotopy trajectory (see F\text{i}gure \ref{pcorrector}). This predictor can be considered as a step of the Euler
method for solving the differential equation that describes the homotopy trajectory (continuation path). 
Predictor steps are usually based on tangent predictions but there are
different alternatives like the secant predictor \cite{homo_seydel}, interpolation predictor
\cite{homo_allgower}, Taylor polynomial predictor \cite{homo_allgower}. 


\begin{figure}[hbtp]
\psfrag{A}{{$(x^j,\lambda_j)$}}
\psfrag{B}{{$(\bar{x}^{j+1},\bar{\lambda}_{j+1})$}}
\psfrag{C}{{$(x^{j+1},\lambda_{j+1})$}}
\centerline{
\epsfxsize=80mm
\epsffile{/home/hleal/papers/doblelim/figs/pcorrector.eps}}
\caption{Predictor-corrector steps}
\label{pcorrector}
\end{figure}

 
\subsubsection{Corrector}

When the predictor step f\text{i}nishes, it is necessary to f\text{i}t the homotopy trajectory by
using a corrector step. This step  solves the Equation \ref{fx} by starting
from $(\bar{x}^{j+1},\bar{\lambda}_{j+1})$ (see F\text{i}gure \ref{pcorrector}). The common
way to solve this equation is by using the Newton-Gauss method \cite{homo_allgower}, which
can solved for systems of the type $R^{N+1} \Rightarrow R^N$.

\subsubsection{Step Control}

The Predictor-Corrector scheme can be optimized for tracing acceleration by
using a step length control. A small constant step length can trace the curve successfully
but not eff\text{i}ciently, because this process involves too many steps along ``f\text{l}at'' branches.
Therefore, it is necessary to adapt the step length to the convergence behavior at each
predictor-corrector step. The basic criterion is to control the step by observing the
convergence quality of the corrector step. A change of the number of iterations in the corrector step produces
a compensation factor $\zeta$, which affects the step length as follows:


\begin{displaymath}
h_{j+2}=h_{j+1} \zeta
\label{zero}
\end{displaymath}


\subsubsection{F\text{i}nd Zero Strategy}

Without an eff\text{i}cient f\text{i}nding zero strategy
the numerical continuation is incomplete and the homotopy could fail to converge some operating
points (maybe not any solution at all).
The tracing of the homotopy trajectory begins in $\lambda=0$ and it ``ends''\footnote{In fact if it is wanted to f\text{i}nd multiple solutions the numerical continuation should follow beyond $\lambda=1$.} when $\rho=1$. Therefore,
when the tracing is near to $\lambda=1$ the {\it f\text{i}nd Zero Strategy} begins. The most simple example
of strategy is to use $[\pig{x}_f,\lambda_f]$ (of the last iteration) like the initial point
to solve the $\pig{f}(\pig{x})$ with Newton method. 


Nevertheless, as Newton method has local convergence it still could fail
to f\text{i}nd the solution. In \cite{homo_sosonkina} are reported some techniques to do the {\it f\text{i}nd Zero Strategy} accurately and
reliable.  The basic idea is
to use two points ($p_1 < 1$ and $p_2 \geq 1$) around to $\lambda=1$, next, interpolate the point in $\lambda=1$ in order to
obtain a point nearly to the real solution and use a like Newton method to f\text{i}nd the solution
of the original system $\pig{f}(\pig{x})$.
The double bounded homotopy has the characteristic of never cross the $\lambda=1$, hence, the f\text{i}ndig zero strategy
should start when  bouncing in the bounding lines. A good way to achieve this process is to monitor the
change of sign of the  $\Delta\lambda$ produced in the predictor step. This can be done by multipplaying
the  $\Delta\lambda$ of two consecutive predictor steps.

\begin{displaymath}
sign(\triangle \lambda_{j+1}  \triangle \lambda_j) \ne -1 
\label{zero1}
\end{displaymath}



\section{Examples}


In order to illustrate the use of the double bounded homotopy, it is  applied to the next system of NAEs:
\begin{displaymath}
\begin{array}{c}
f_1(x_1,x_2)=(x_2-1)(x_2-4)(x_2-6)+x_1=0\\
f_2(x_1,x_2)=(x_1-3)(x_1-6)(x_1-9)+x_2=0
\end{array}
\end{displaymath}
The graphic solution of the system is shown in F\text{i}gure \ref{9sol}.

\begin{figure}[hbtp]
\centerline{
\epsfxsize=90mm
\epsffile{/home/hleal/papers/doblelim/figs/doblelimit_mul.eps}}
\caption{System of f\text{i}ve solutions}
\label{9sol}
\end{figure}

The homotopy formulation from Equation \ref{fx} results in:
\begin{displaymath}
\begin{array}{c}
H_1(f_1,\lambda)=100Q+e^{Q}\ln(0.001f_1^2+1)=0\\
H_2(f_2,\lambda)=100Q+e^{Q}\ln(0.001f_2^2+1)=0\\
\end{array}
\end{displaymath}
where $Q=\lambda(\lambda-1)$; i.e. $a=0$ and $b=1$.



The homotopy trajectories  are depicted
in F\text{i}gure \ref{homote}. The starting points lie on the plane def\text{i}ned by $\lambda=0.5$, while,
the solutions are obtained when $\lambda$ reaches the value of 1.

\begin{figure}[h]
\psfrag{ll}{$\lambda=0.5$}
\psfrag{l}{$\lambda$}
\psfrag{l1}{$\lambda=1$}
\psfrag{x}{$x_1$}
\psfrag{y}{$x_2$}
\centerline{
\epsfxsize=90mm
\epsffile{/home/hleal/papers/doblelim/figs/doblelimit_mul_sol.eps}}
\caption{Homotopy trajectory}
\label{homote}
\end{figure}




The double bounded homotopy is applied to the latch circuit \cite{homo_dobletrazado}  of
F\text{i}gure  \ref{circu2}, which contains
two NMOS transistors ($M_1$ and $M_2$), two linear resistors
($R_1$ and $R_2$) and a voltage source ($E$). The model of the transistors is the unif\text{i}ed MOS model
reported in \cite{homo_BSIM} which is a modif\text{i}ed version of the well-known
BSIM model.


\begin{figure}[hbtp]
\centerline{
\epsfxsize=90mm
\epsffile{/home/hleal/papers/doblelim/figs/circu.eps}
}
\caption{Example circuit}
\label{circu2}
\end{figure}

The  F\text{i}gure \ref{circugraf} show the graph of the equilibrium equation and homotopy trajectory of
the circuit in the space of $(v_1,v_2,I_E)$. The homotopy f\text{i}nd the three solutions of the circuit.

\begin{figure}[hbtp]
\centerline{
\epsfxsize=90mm
\epsffile{/home/hleal/papers/doblelim/figs/smitt.eps}
}
\caption{Graphic of the equilibrium equation of the circuit and homotopy trajectory}
\label{circugraf}
\end{figure}





F\text{i}gure \ref{chua} depict the chua's benchmark circuit of nine solutions. This circuit has 4 bipolar transistors modeled by the half-side Ebers-Moll model.

The equations system formulation is the same of \cite{homo_chua} which is based
on the brach voltages $v_1$, $v_2$, $v_3$, $v_4$. F\text{i}gure \ref{chuaf} depict
the homotopy trajectory and the six solutions found versus the branch voltage $v_1$.

\begin{figure}[hbt]
\centerline{
\epsfxsize=90mm
\epsffile{/home/hleal/papers/doblelim/figs/chiaf.eps}}
\caption{Solution of the Chua's circuit}
\label{chuaf}
\end{figure}


\begin{figure}[hbt]
\psfrag{F}{{\tiny $I_D=10^{-9}(e^{40v_D}-1)$}}
\psfrag{b}{{\tiny $\alpha=0.98$}}
\centerline{
\epsfxsize=90mm
\epsffile{/home/hleal/papers/iscas2004/figs/chua.eps}}
\caption{Chua's circuit with nine solutions}
\label{chua}
\end{figure}

\section{Conclusions}
In this work presents a numerical continuation for tracing a new homotopy
method to obtain the operating points of nonlinear circuits having
multiple solutions. 
This homotopy method is called the double bounded homotopy and it overcomes the shortcomings of
traditional homotopy schemes regarding the stop criterion. The geometrical properties of the method have been
analysed by using the Lambert $\cal W$ function. The key aspects of the implementation of the numerical continuation were depicted.

%\newpage
\bibliographystyle{IEEEbib}
 \bibliography{/home/hleal/papers/doblelim/bib/homoto,/home/hleal/papers/doblelim/bib/contin,/home/hleal/papers/doblelim/bib/netwth}



%\begin{abstract}
%The abstract goes here.
%\end{abstract}

% no keywords

% For peer review papers, you can put extra information on the cover
% page as needed:
% \begin{center} \bfseries EDICS Category: 3-BBND \end{center}
%
% for peerreview papers, inserts a page break and creates the second title.
% Will be ignored for other modes.
\IEEEpeerreviewmaketitle



%\section{Introduction}
% no \PARstart
%This demo file is intended to serve as a ``starter file"
%for IEEE conference papers produced under \LaTeX\ using IEEEtran.cls version
%1.6b and later.

% May all your publication endeavors be successful.

%\hfill mds
 
%\hfill November 18, 2002


%\subsection{Subsection Heading Here}
%Subsection text here.

%\subsubsection{Subsubsection Heading Here}
%Subsubsection text here.

% Reminder: the "draftcls" or "draftclsnofoot", not "draft", class option
% should be used if it is desired that the figures are to be displayed while
% in draft mode.

% An example of a floating figure using the graphicx package.
% Note that \label must occur AFTER (or within) \caption.
% For figures, \caption should occur after the \includegraphics.
%
%\begin{figure}
%\centering
%\includegraphics[width=2.5in]{myfigure}
% where an .eps filename suffix will be assumed under latex, 
% and a .pdf suffix will be assumed for pdflatex
%\caption{Simulation Results}
%\label{fig_sim}
%\end{figure}


% An example of a double column floating figure using two subfigures.
%(The subfigure.sty package must be loaded for this to work.)
% The subfigure \label commands are set within each subfigure command, the
% \label for the overall fgure must come after \caption.
% \hfil must be used as a separator to get equal spacing
%
%\begin{figure*}
%\centerline{\subfigure[Case I]{\includegraphics[width=2.5in]{subfigcase1}
% where an .eps filename suffix will be assumed under latex, 
% and a .pdf suffix will be assumed for pdflatex
%\label{fig_first_case}}
%\hfil
%\subfigure[Case II]{\includegraphics[width=2.5in]{subfigcase2}
% where an .eps filename suffix will be assumed under latex, 
% and a .pdf suffix will be assumed for pdflatex
%\label{fig_second_case}}}
%\caption{Simulation results}
%\label{fig_sim}
%\end{figure*}



% An example of a floating table. Note that, for IEEE style tables, the 
% \caption command should come BEFORE the table. Table text will default to
% \footnotesize as IEEE normally uses this smaller font for tables.
% The \label must come after \caption as always.
%
%\begin{table}
%% increase table row spacing, adjust to taste
%\renewcommand{\arraystretch}{1.3}
%\caption{An Example of a Table}
%\label{table_example}
%\begin{center}
%% Some packages, such as MDW tools, offer better commands for making tables
%% than the plain LaTeX2e tabular which is used here.
%\begin{tabular}{|c||c|}
%\hline
%One & Two\\
%\hline
%Three & Four\\
%\hline
%\end{tabular}
%\end{center}
%\end{table}


%\section{Conclusion}
%The conclusion goes here.

% conference papers do not normally have an appendix

% use section* for acknowledgement
%\section*{Acknowledgment}
% optional entry into table of contents (if used)
%\addcontentsline{toc}{section}{Acknowledgment}
%The authors would like to thank...

% trigger a \newpage just before the given reference
% number - used to balance the columns on the last page
% adjust value as needed - may need to be readjusted if
% the document is modified later
%\IEEEtriggeratref{8}
% The "triggered" command can be changed if desired:
%\IEEEtriggercmd{\enlargethispage{-5in}}

% references section
% NOTE: BibTeX documentation can be easily obtained at:
% http://www.ctan.org/tex-archive/biblio/bibtex/contrib/doc/

% can use a bibliography generated by BibTeX as a .bbl file
% standard IEEE bibliography style from:
% http://www.ctan.org/tex-archive/macros/latex/contrib/supported/IEEEtran/bibtex
%\bibliographystyle{IEEEtran.bst}
% argument is your BibTeX string definitions and bibliography database(s)
%\bibliography{IEEEabrv,../bib/paper}
%
% <OR> manually copy in the resultant .bbl file
% set second argument of \begin to the number of references
% (used to reserve space for the reference number labels box)
%\begin{thebibliography}{1}

%\bibitem{IEEEhowto:kopka}
%H.~Kopka and P.~W. Daly, \emph{A Guide to {\LaTeX}}, 3rd~ed.\hskip 1em plus
%  0.5em minus 0.4em\relax Harlow, England: Addison-Wesley, 1999.

%\end{thebibliography}


% that's all folks
\end{document}


