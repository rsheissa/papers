\documentclass[conference]{IEEEtran}
\hyphenation{op-tical net-works semi-conduc-tor IEEEtran}




%
%	LINear ALGebra Macros
%
%  a) Matrix & vector alike manipulations
%
%

%%BUT firstly, a macro to set the first letter and word of
%%a paagraph in a bigger size than the rest of the text.
%% arg1=1stletter, arg2=rest_of_word
%% Example
%% /*****\HIS is a big T.
%%    *
%%    *
\def\PARAstart#1#2{\begingroup\def\par{\endgraf\endgroup\lineskiplimit=0pt}
    \setbox2=\hbox{\uppercase{#2} }\newdimen\tmpht \tmpht \ht2
    \advance\tmpht by \baselineskip\font\hhuge=cmr10 at \tmpht
    \setbox1=\hbox{{\hhuge #1}}
    \count7=\tmpht \count8=\ht1\divide\count8 by 1000 \divide\count7 by\count8
    \tmpht=.001\tmpht\multiply\tmpht by \count7\font\hhuge=cmr10 at \tmpht
    \setbox1=\hbox{{\hhuge #1}} \noindent \hangindent1.05\wd1
    \hangafter=-2 {\hskip-\hangindent \lower1\ht1\hbox{\raise1.0\ht2\copy1}%
    \kern-0\wd1}\copy2\lineskiplimit=-1000pt}


%
%	HORIZONTAL VECTORS	x1, x2, ... xn

%	x1, x2, .. xn
%
\newcommand{\hvec}
	   {\mbox{$x_1, x_2,\cdots,x_n$ }}

%	(x)1, (x)2, ... (x)n
%
\newcommand{\hvecx}[1]
	   {\mbox{${#1}_1, {#1}_2,\cdots,{#1}_n$ }}

%
%	(x)1, (y)2, ... (z)n
\newcommand{\hvecxyz}[3]
	   {\mbox{${#1}_1, {#2}_2,\cdots,{#3}_n$ }}

%
%	VERTICAL VECTORS

\newcommand{\vvec}
		{\mbox{$
			\begin{array}{c}
			x_1 \\ x_2 \\ \vdots \\ x_n \\
			\end{array}
		$}}

%
\newcommand{\vvecx}[1]
		{\mbox{$
			\begin{array}{c}
			#1_1 \\ #1_2 \\ \vdots \\ #1_n \\
			\end{array}
		$}}

%
\newcommand{\vvecxyz}[3]
		{\mbox{$
			\begin{array}{c}
			#1_1 \\ #2_2 \\ \vdots \\ #3_n \\
			\end{array}
		$}}

%
\newcommand{\dotcol}
{\mbox{$ 
	\begin{array}{c}
		\cdots\vspace{3mm}\\
		\cdots\vspace{3mm}\\
		\cdots\vspace{3mm}\\
		\cdots\vspace{3mm}\\
	\end{array}
$}}


%
%	GENERIC MATRICES


%
\newcommand{\genmtrx}[3]
{\mbox{$	%\left[
\pmatrix{
	#1_{11}	   & #1_{12}	& \cdots & #1_{1{#2}}	\cr 
	#1_{21}	   & #1_{22}	& \cdots & #1_{2{#2}}	\cr 
	\vdots 	   & \vdots 	& \ddots & \vdots	\cr
	#1_{{#3}1} & #1_{{#3}2} & \cdots & #1_{{#3}{#2}}\cr 
		%\rigth]
}
	$}}

%

%
				%% [ p1_{p2 p3} ]
\newcommand{\biggmtrx}[3]
{\mbox{$
	\left[ 
	\begin{array}{c}
		{#1}_{11}	     \vspace{3mm}\\ 
		{#1}_{21}	     \vspace{3mm}\\
		\vdots	     \vspace{3mm}\\ 
		{#1}_{{#3}1}   \vspace{3mm}\\ 
	\end{array}
	\begin{array}{c}
		{#1}_{12}      \vspace{3mm}\\ 
		{#1}_{22}      \vspace{3mm}\\
		\vdots       \vspace{3mm}\\ 
		{#1}_{{#3}2}   \vspace{3mm}\\ 
	\end{array}
	\begin{array}{c}
		\cdots\vspace{3mm}\\
		\cdots\vspace{3mm}\\
		\ddots\vspace{3mm}\\
		\cdots
	\end{array}
	\begin{array}{c}
		{#1}_{1{#2}}         \vspace{3mm}\\ 
		{#1}_{2{#2}}         \vspace{3mm}\\
		\vdots            \vspace{3mm}\\
		{#1}_{{#3}{#2}}     \vspace{3mm}\\
	\end{array}
	\right]
$}}


%
% List of macros defining
% Some Jacobians and matrices used for numerical
% items.
%


			% | d p1_p2 |
\newcommand{\dpart}[4]	%$|---------|$
			% | d p3_p4 |
{\mbox{$
	\frac{\partial {#1}_{#2} }
	     {\partial {#3}_{#4} }
      $}
}

\newcommand{\bigjac}[2]
{\mbox{$
	\left[ 
	\begin{array}{c}
		\dpart{#1}{1}{#2}{1} \vspace{3mm}\\ 
		\dpart{#1}{2}{#2}{1} \vspace{3mm}\\
		\vdots \vspace{3mm}\\ 
		\dpart{#1}{n}{#2}{1} \vspace{3mm}\\ 
	\end{array}
	\begin{array}{c}
		\dpart{#1}{1}{#2}{2} \vspace{3mm}\\ 
		\dpart{#1}{2}{#2}{2} \vspace{3mm}\\
		\vdots \vspace{3mm}\\
		\dpart{#1}{n}{#2}{2} \vspace{3mm}\\ 
	\end{array}
	\begin{array}{c}
		\cdots\vspace{3mm}\\
		\cdots\vspace{3mm}\\
		\ddots\vspace{3mm}\\
		\cdots
	\end{array}
	\begin{array}{c}
		\dpart{#1}{1}{#2}{n} \vspace{3mm}\\ 
		\dpart{#1}{2}{#2}{n} \vspace{3mm}\\
		\vdots \vspace{3mm} \\
		\dpart{#1}{n}{#2}{n} \vspace{3mm}\\
	\end{array}
	\right]
$}}


\newcommand{\chuajac}[3]
{\mbox{$
	\left[ 
	\begin{array}{c}
		\dpart{#1}{1}{#2}{1} \vspace{3mm}\\ 
		\dpart{#1}{2}{#2}{1} \vspace{3mm}\\
		\vdots \vspace{3mm}\\ 
		\dpart{#1}{n}{#2}{1} \vspace{3mm}\\ 
		0
	\end{array}
	\begin{array}{c}
		\dpart{#1}{1}{#2}{2} \vspace{3mm}\\ 
		\dpart{#1}{2}{#2}{2} \vspace{3mm}\\
		\vdots \vspace{3mm}\\
		\dpart{#1}{n}{#2}{2} \vspace{3mm}\\ 
		0
	\end{array}
	\begin{array}{c}
		\cdots\vspace{3mm}\\
		\cdots\vspace{3mm}\\
		\vdots\vspace{3mm}\\
		\cdots\vspace{3mm}\\
		\cdots
	\end{array}
	\begin{array}{c}
		\dpart{#1}{1}{#2}{k} \vspace{3mm}\\ 
		\dpart{#1}{2}{#2}{k} \vspace{3mm}\\
		\vdots \vspace{3mm} \\
		\dpart{#1}{n}{#2}{k} \vspace{3mm}\\
		sgn(\frac{dx_k}{dp})
	\end{array}
	\begin{array}{c}
		\cdots\vspace{3mm}\\
		\cdots\vspace{3mm}\\
		\ddots\vspace{3mm}\\
		\cdots\vspace{3mm}\\
		\cdots
	\end{array}
	\begin{array}{c}
		\dpart{#1}{1}{#2}{n} \vspace{3mm}\\ 
		\dpart{#1}{2}{#2}{n} \vspace{3mm}\\
		\vdots \vspace{3mm} \\
		\dpart{#1}{n}{#2}{n} \vspace{3mm}\\
		0
	\end{array}
	\begin{array}{c}
		\dpart{#1}{1}{#3}{} \vspace{3mm}\\ 
		\dpart{#1}{2}{#3}{} \vspace{3mm}\\
		\vdots \vspace{3mm}\\
		\dpart{#1}{n}{#3}{} \vspace{3mm}\\
		0
	\end{array}
	\right]
$}}
%==============
%==============

\newcommand{\jacob}[2]
			{\mbox{$ \left[
			{\mbox{$ 
				{\coljac}{#1}{#2}{1}
			$}}
			{\mbox{$ 
				{\coljac}{#1}{#2}{2}
			$}}
			{\mbox{$ 
				{\dotcol}
			$}}
			{\mbox{$ 
				{\coljac}{#1}{#2}{n}
			$}}

			\rigth] $}}



\newcommand{\jac}[2]
{\mbox{$	%\left[
\pmatrix{
{\frac{\partial #1_1}{\partial #2_1} }& 
	{\frac{\partial #1_1}{\partial #2_2} }& 
		\cdots &
			{\frac{\partial #1_1}{\partial #2_n} }\cr 
{\frac{\partial #1_2}{\partial #2_1} }& 
	{\frac{\partial #1_2}{\partial #2_2} }& 
		\cdots &
			{\frac{\partial #1_2}{\partial #2_n} }\cr 
\vdots& \vdots& \ddots& \vdots\cr
{\frac{\partial #1_n}{\partial #2_1} }& 
	{\frac{\partial #1_n}{\partial #2_2} }& 
		\cdots &
			{\frac{\partial #1_n}{\partial #2_n} }\cr
		%\rigth]
}
	$}}


\newcommand{\coljac}[3]
{\mbox{$ 
	\begin{array}{c}
		\frac{\partial #1_1}{\partial #2_#3}\vspace{3mm} \\ 
		\frac{\partial #1_2}{\partial #2_#3}\vspace{3mm} \\ 
		\vdots \vspace{3mm} \\
		\frac{\partial #1_n}{\partial #2_#3}\vspace{3mm} \\ 
	\end{array}
$}}

%
%	C and O subset alike macros
% \cmd = normal comand
% \mcmd = comand surrounded by $ ... $
%

%
% C(T, B)
\newcommand{\cset}{\cal{C}(\pig{T}, \pig{B})}
\newcommand{\oset}{\cal{O}(\pig{T}, \pig{B})}
\newcommand{\mcset}{\mbox{$\cal{C}(\pig{T}, \pig{B})$} }
\newcommand{\moset}{\mbox{$\cal{O}(\pig{T}, \pig{B})$} }

\newcommand{\obe}{{\cal{O}}^{be}}
\newcommand{\mobe}{\mbox{${\cal{O}}^{be}$ }}
\newcommand{\omix}{{\cal{O}}^{mix}}
\newcommand{\momix}{\mbox{${\cal{O}}^{mix}$ }}
\newcommand{\obc}{{\cal{O}}^{bc}}
\newcommand{\mobc}{\mbox{${\cal{O}}^{bc}$ }}

%\newcommand{\pig}[1]{\mbox{\boldmath ${#1}$}	}




%\usepackage{spconf}
\newtheorem{Theod}{{\bf Def\text{i}nition}}
\newtheorem{Theos}{{\bf Statement}}
\newtheorem{Theor}{{\bf Theorem}}
\newtheorem{Theoc}{{\bf Corallary}}

\usepackage[dvips]{epsfig}
\usepackage{epsfig}
\usepackage{psfrag}
\usepackage{pifont}
\usepackage{amssymb}
\usepackage{amsmath}
\usepackage{rotate}
\usepackage{rotating}




\begin{document}

% paper title
\title{Numerical continuation scheme for tracing the double bounded homotopy for analysing  nonlinear circuits}

\author{H\'ector V\'azquez-Leal, Luis Hern\'andez-Mart\'inez, Arturo Sarmiento-Reyes and Roberto Casta\~neda-Sheissa \\ National Institute for Astrophysics, Optics and Electronics\\
Electronics Department, CAD Group\\ P.O. Box 51, 72000, Puebla, Pue., Mexico \\ E-mail {\it  hleal@inaoep.mx}}


% avoiding spaces at the end of the author lines is not a problem with
% conference papers because we don't use \thanks or \IEEEmembership


% for over three affiliations, or if they all won't fit within the width
% of the page, use this alternative format:
% 
%\author{\authorblockN{Michael Shell\authorrefmark{1},
%Homer Simpson\authorrefmark{2},
%James Kirk\authorrefmark{3}, 
%Montgomery Scott\authorrefmark{3} and
%Eldon Tyrell\authorrefmark{4}}
%\authorblockA{\authorrefmark{1}School of Electrical and Computer Engineering\\
%Georgia Institute of Technology,
%Atlanta, Georgia 30332--0250\\ Email: mshell@ece.gatech.edu}
%\authorblockA{\authorrefmark{2}Twentieth Century Fox, Springfield, USA\\
%Email: homer@thesimpsons.com}
%\authorblockA{\authorrefmark{3}Starfleet Academy, San Francisco, California 96678-2391\\
%Telephone: (800) 555--1212, Fax: (888) 555--1212}
%\authorblockA{\authorrefmark{4}Tyrell Inc., 123 Replicant Street, Los Angeles, California 90210--4321}}



% use only for invited papers
%\specialpapernotice{(Invited Paper)}

% make the title area
\maketitle



\begin{abstract}
A numerical continuation for tracing the double bounded homotopy for obtaining  DC solutions of 
nonlinear circuits is proposed.
The {\it double bounded homotopy} is used  to f\text{i}nd multiple DC solutions 
with the advantage of having
a stop criterion  which is based on the property of having a double bounded trajectory. 
The key aspects of the implementation of the numerical continuation are presented in this paper. Besides, in order to
trace and apply the stop criterion some blocks of the numerical continuation are modif\text{i}ed and explained.
\end{abstract}


\section{Introduction}

Homotopy methods \cite{homo_ArtificialP}\cite{homo_DWolfMulti}\cite{cont_leu1} have an important role in the analysis of
circuits exhibiting multiple operating points.
Although, homotopy methods are able to f\text{i}nd more than one solution to
the equilibrium equation of the circuit, they still show several problems. Among them, it is worthy to mention the {\bf stop criterion}. 

It is well known that there are two types of paths of solutions,
open and closed paths, the main problem is when to stop searching for
more solutions. For closed paths, this can be solved by testing whether
a new solution is not indeed a previesly found solution. For open paths, this is a serious
drawback, because there is no reasonable and reliable stop criterion
to decide when to stop seeking for more solutions.

The Double Bounded Homotopy (DBH) \cite{homo_iscas05}\cite{homo_iberchip03} has been proposed in order to circumvent the problem
of the stop criterion. This homotopy possesses symmetrical branches bounded by a double bounding solution line. The symmetry and
bounding properties of the trajectory of the DBH are depicted in F\text{i}gure \ref{halftrack}. These properties are
useful in order to implement a reliable stop criterion. 

The numerical continuation methods (also called path following and path tracking) are numerical tools used to
trace the homotopy trajectories. They are a combination of a variety of numerical methods, focussed on drawing 
a path in order to accomplish specif\text{i}c needs of a particular homotopy formulation. 

Due to the specific features of the DBH, it becomes necessary to devise a well-suited numerical continuation method
in order to trace the homotopy trajectory having  a robust stop criterion.
The next section is dedicate to explain the traditional numerical continuation methods.

\begin{figure}[hbtp]
\psfrag{xi}{{\small $[x_i,(a+b)/2]$}}
\psfrag{xf}{{\small $[x_f,(a+b)/2]$}}
\psfrag{o}{$\lambda$}
\centerline{
\epsfxsize=62mm
\epsffile{/home/hleal/papers/icccas/Template_LaTex/figs/dbh.eps}}
\caption{Symmetry and bounding of DBH}
\label{halftrack}
\end{figure}


\section{Numerical continuation methods}

Some homotopies have global convergence when are applied to solve the equilibrium equation of certain type of circuits. 
However, without an appropriated numerical continuation method is not possible to ensure that all the solutions of
the equilibrium equation will be found.  There are some reasons for
this problem, one of them is related with the
predictor-corrector steps, if the coeff\text{i}cients of these steps are not properly selected,
then the numerical continuation fails and losts
the homotopy trajectory. The other reason is that once the numerical continuation crosses the solution line,
the algorithm fails to determine the solution because it diverges. It clearly results that it is important to study the characteristics
of the numerical continuation in order to use them appropriately.
The numerical continuation scheme consists of a predictor, a corrector, a step control, a f\text{i}nd zero strategy and
a stop criterion.

\newpage

\subsubsection{Predictor}

The predictor point for $(x^j,\lambda_j)$ is given by:

\begin{displaymath}
(\bar{\pig{x}}^{j+1},\bar{\lambda}_{j+1})=(\pig{x}^j,\lambda_j)+h*\pig{t}
\end{displaymath}
where $h$ is an appropriate step length and $\pig{t}$ is a normalised tangent vector to the
homotopy trajectory (see F\text{i}gure \ref{pcorrector}). This predictor can be considered as a step of the Euler
method  (or some numerical integration method) for solving the differential equation that describes
the homotopy trajectory (continuation path). 
Predictor steps are usually based on tangent predictions but there are
several alternatives like
 the secant predictor \cite{homo_seydel}, interpolation predictor
\cite{homo_allgower}, Taylor polynomial predictor \cite{homo_allgower}. 


\begin{figure}[hbtp]
\psfrag{A}{{$(x^j,\lambda_j)$}}
\psfrag{B}{{$(\bar{x}^{j+1},\bar{\lambda}_{j+1})$}}
\psfrag{C}{{$(x^{j+1},\lambda_{j+1})$}}
\centerline{
\epsfxsize=60mm
\epsffile{/home/hleal/papers/doblelim/figs/pcorrector.eps}}
\caption{Predictor-corrector steps}
\label{pcorrector}
\end{figure}

 
\subsubsection{Corrector}

When the predictor step f\text{i}nishes, it is necessary to f\text{i}t the homotopy trajectory by
using a corrector step. This step  solves the homotopy formulation by starting
from $(\bar{x}^{j+1},\bar{\lambda}_{j+1})$ (see F\text{i}gure \ref{pcorrector}). The common
way to solve this equation is by using the Newton-Gauss method \cite{homo_allgower}, which
can be solved for systems of the type $R^{N+1} \Rightarrow R^N$.

\subsubsection{Step Control}

The Predictor-Corrector scheme can be optimized for tracing acceleration by
using a step length control. A small constant step length can trace the curve successfully
but not eff\text{i}ciently, because this process involves too many steps along ``f\text{l}at'' branches.
Therefore, it is necessary to adapt the step length to the convergence behavior at each
predictor-corrector step. The basic criterion is to control the step by observing the
convergence quality of the corrector step. A change on the number of iterations in the corrector step produces
a compensation factor $\zeta$, which affects the step length as follows:


\begin{displaymath}
h_{j+2}=h_{j+1} \zeta
\label{zero}
\end{displaymath}


\subsubsection{F\text{i}nd Zero Strategy}

Without an eff\text{i}cient f\text{i}nding zero strategy
the numerical continuation is incomplete and the homotopy could fail
to converge to some solutions.
The tracing of the homotopy trajectory begins at $\lambda=0$ and it ends\footnote{In fact if it is wanted to f\text{i}nd multiple solutions the numerical continuation
should follow beyond $\lambda=1$.} at $\lambda=1$. When the tracing is close to $\lambda=1$, the {\it f\text{i}nd Zero Strategy} starts. The simplest example
of strategy is to use $[\pig{x}_f,\lambda_f]$ (the last iteration) as the initial point
to solve the equilibrium equation $\pig{f}(\pig{x})$ with a Newton-like method. 


Because the Newton method possesses local convergence, it still could fail
to f\text{i}nd the solution. In \cite{homo_sosonkina} some techniques are reported
that implement the {\it f\text{i}nd Zero Strategy} accurately and
reliably.  The basic idea is
to use two points ($p_1 < 1$ and $p_2 \geq 1$) around to $\lambda=1$, and interpolate the point at $\lambda=1$ in order to
obtain a point closely to the real solution and use a Newton-like method to f\text{i}nd the solution
of the original system $\pig{f}(\pig{x})$.

\subsubsection{Stop criterion}

In fact, there is not stop criterion in the traditional numerical continuation methods when applied to homotopy trajectory tracing.
The most common way to stop tracing the trajectory  is to set a maximum allowed number ($ITMAX$) of predictor-corrector steps without f\text{i}nding
any solution. This technique is ineff\text{i}cient because it can fail
to f\text{i}nd some solutions on the homotopy trajectory.

All the blocks of the numerical continuation method are depicted in F\text{i}gure \ref{bloquesf}. 





\begin{figure}[hbtp]
\psfrag{p}{{$(\bar{x}^{j+1},\bar{\lambda}_{j+1})$}}
\psfrag{c}{{$(x^{j+1},\lambda_{j+1})$}}
\psfrag{s}{{$h_{j+2}=h_{j+1}\zeta$}}
\psfrag{l1}{{\tiny $\lambda=1?$}}
\psfrag{l1b}{{$  \quad or \qquad   \lambda_{j} \ge 1   \quad and   \quad   \lambda_{j+1} \le 1  $}}
\psfrag{l2}{{\small $k<k^*$}}
\psfrag{l}{{\small $\lambda=0$}}
\centerline{
\epsfxsize=46mm
\epsffile{/home/hleal/papers/icccas/Template_LaTex/figs/bloquesf.eps}}
\caption{Numerical continuation scheme}
\label{bloquesf}
\end{figure}


\section{Modif\text{i}ied Numerical continuation}

This section explains the modif\text{i}cations accomplished on the
scheme above with the idea of providing a reliable stop criterion
to the DBH. The modifications act on both the Find Zero Strategy and
the stop criterion itself.

\begin{enumerate}
\item { \bf F\text{i}nd Zero Strategy}. The double bounded homotopy
has the characteristic of never crossing $\lambda=1$ \cite{homo_iscas05}, hence, the f\text{i}ndig zero strategy
should starts after the trajectory   bounces on the bounding line. An
efficient way to achieve this process is by monitoring the
change of sign of $\Delta\lambda$ produced in the predictor step. This
can be done by multiplying
$\Delta\lambda$ of two consecutive predictor steps. 
\begin{displaymath}
sign(\triangle \lambda_{j+1}  \triangle \lambda_j) \ne -1 
\label{zero1}
\end{displaymath}
This procedure is depicted in F\text{i}gure \ref{halftrack3}, where the  sign of  $\Delta\lambda$ changes in the bouncing  points $A$ and $B$. Besides, in order to apply a quadratic interpolation
the algorithm needs three points ($A,B,C$).

\begin{figure}[hbtp]
\psfrag{xi}{{\small $[x_i,(a+b)/2]$}}
\psfrag{xf}{{\small $[x_f,(a+b)/2]$}}
\psfrag{o}{$\lambda$}
\centerline{
\epsfxsize=50mm
\epsffile{/home/hleal/papers/icccas/Template_LaTex/figs/dbh3.eps}}
\caption{F\text{i}nd Zero Strategy}
\label{halftrack3}
\end{figure}
\item{\bf Stop criterion}. The stop criterion for this homotopy is depicted in F\text{i}gure \ref{halftrack1}. The homotopy trajectory starts at the symmetry axis of
the homotopy trajectory. Then it traces the half of the trayectory (the symmetrical branch) and stops when it returns to the symmetry axis.
The advantage of the stop criterion is that the numerical continuation
needs only to trace the half side of the trayectory.
\begin{figure}[hbtp]
\psfrag{xi}{{\small $[x_i,(a+b)/2]$}}
\psfrag{xf}{{\small $[x_f,(a+b)/2]$}}
\psfrag{o}{$\lambda$}
\centerline{
\epsfxsize=55mm
\epsffile{/home/hleal/papers/icccas/Template_LaTex/figs/dbh2.eps}}
\caption{Stop criterion}
\label{halftrack1}
\end{figure}
\end{enumerate}

The modif\text{i}ed numerical continuation is depicted in F\text{i}gure \ref{bloques}
where the dashed blocks are the specif\text{i}c characteristics added to the procedure.
The scheme is explained as follows: It starts when the predictor  calculates the tangent to
$(x^j,\lambda_j)$ and using a step length calculates the $(\bar{x}^{j+1},\bar{\lambda}_{j+1})$
over the tangent. Then, the corrector uses the solution of the predictor in order to obtain
a new point on the homotopy trajectory, given by  $(x^{j+1},\lambda_{j+1})$.
At this point, the step control is applied in order to accelerate the tracing.
Next, the so-called end-zero strategy is applied which is 
triggered at each solution when the trajectory bounces on
the solution line.
F\text{i}nally, the numerical continuation stops tracing when the trajectory returns to the symmetry axis.


\begin{figure}[hbtp]
\psfrag{p}{{$(\bar{x}^{j+1},\bar{\lambda}_{j+1})$}}
\psfrag{c}{{$(x^{j+1},\lambda_{j+1})$}}
\psfrag{s}{{$h_{j+2}=h_{j+1}\zeta$}}
\psfrag{l1}{{\small $sign(\triangle \lambda_{j+1}  \triangle \lambda_j) \ne -1 $}}
\psfrag{l2}{{\small $ \lambda \ne (a+b)/2$}}
\psfrag{lf}{{\small $\lambda$}}
\centerline{
\epsfxsize=48mm
\epsffile{/home/hleal/papers/icccas/Template_LaTex/figs/bboques.eps}}
\caption{Block diagram of the modif\text{i}ed numerical continuation}
\label{bloques}
\end{figure}


\section{Examples}


In order to illustrate the use of the DBH with the modifications, a
first example is used to solve the system of equations given as:
\begin{displaymath}
\begin{array}{c}
f_1(x_1,x_2)=(x_2-1)(x_2-4)(x_2-6)+x_1=0\\
f_2(x_1,x_2)=(x_1-3)(x_1-6)(x_1-9)+x_2=0
\end{array}
\end{displaymath}
The graphic solution of the system is shown in F\text{i}gure \ref{9sol}.

\begin{figure}[hbtp]
\centerline{
\epsfxsize=90mm
\epsffile{/home/hleal/papers/doblelim/figs/doblelimit_mul.eps}}
\caption{System of f\text{i}ve solutions}
\label{9sol}
\end{figure}

The DBH formulation yields:
\begin{displaymath}
\begin{array}{c}
H_1(f_1,\lambda)=100Q+e^{Q}\ln(0.001f_1^2+1)=0\\
H_2(f_2,\lambda)=100Q+e^{Q}\ln(0.001f_2^2+1)=0\\
\end{array}
\end{displaymath}
where $Q=\lambda(\lambda-1)$; i.e. $a=0$ and $b=1$.



The homotopy trajectories  are depicted
in F\text{i}gure \ref{homote}. The starting points lie on the plane def\text{i}ned by $\lambda=0.5$, while,
the solutions are obtained when $\lambda$ reaches the value of 1.

\begin{figure}[h]
\psfrag{ll}{$\lambda=0.5$}
\psfrag{l}{$\lambda$}
\psfrag{l1}{$\lambda=1$}
\psfrag{x}{$x_1$}
\psfrag{y}{$x_2$}
\centerline{
\epsfxsize=90mm
\epsffile{/home/hleal/papers/doblelim/figs/doblelimit_mul_sol.eps}}
\caption{Homotopy trajectory}
\label{homote}
\end{figure}

 A second example is given by the latch circuit of F\text{i}gure
 \ref{circu2}, which  contains
two NMOS transistors ($M_1$ and $M_2$), two linear resistors
($R_1$ and $R_2$) and a voltage source ($E$). The model of the transistors is the unif\text{i}ed MOS model
reported in \cite{homo_BSIM} which is a modif\text{i}ed version of the well-known
BSIM model.


\begin{figure}[hbtp]
\centerline{
\epsfxsize=90mm
\epsffile{/home/hleal/papers/doblelim/figs/circu.eps}
}
\caption{Example circuit}
\label{circu2}
\end{figure}

The  F\text{i}gure \ref{circugraf} shows the graph of the equilibrium
equation and the homotopy trajectory of
the circuit in the space $(v_1,v_2,I_E)$. The homotopy f\text{i}nds
all three solutions of the circuit.

\begin{figure}[hbtp]
\centerline{
\epsfxsize=90mm
\epsffile{/home/hleal/papers/doblelim/figs/smitt.eps}
}
\caption{Graphic of the equilibrium equation of the circuit and homotopy trajectory}
\label{circugraf}
\end{figure}



A last example is the well-known benchmark circuit reported in
\cite{homo_chua}. This circuit has 4 bipolar transistors modeled by the half-sided Ebers-Moll model.

The equations system formulation is the same of \cite{homo_chua} which is based
on the branch voltages $v_1$, $v_2$, $v_3$, $v_4$. F\text{i}gure \ref{chuaf} depicts
the homotopy trajectory and the six solutions found versus $v_1$.

\begin{figure}[hbt]
\centerline{
\epsfxsize=90mm
\epsffile{/home/hleal/papers/doblelim/figs/chiaf.eps}}
\caption{Solution of the Chua's circuit}
\label{chuaf}
\end{figure}


\begin{figure}[hbt]
\psfrag{F}{{\tiny $I_D=10^{-9}(e^{40v_D}-1)$}}
\psfrag{b}{{\tiny $\alpha=0.98$}}
\centerline{
\epsfxsize=90mm
\epsffile{/home/hleal/papers/iscas2004/figs/chua.eps}}
\caption{Chua's circuit}
\label{chua}
\end{figure}


\section{Conclusions}



A numerical continuation for tracing a double bounded homotopy has
been presented. The numerical continuation scheme exhibits an improved
performance regarding stop criterion and finding zero strategy.
Some examples illustrating the aplication of the scheme to nonlinear
resistive circuits were also presented.



\section*{Acknowledgements}
Hector Vazquez Leal is holder of a scholarship from CONACyT M\'exico under contract 143907. This work has been partially supported by a CONACyT M\'exico research project under grant 42588-Y.



%\newpage
\bibliographystyle{IEEEbib}
 \bibliography{/home/hleal/papers/doblelim/bib/homoto,/home/hleal/papers/doblelim/bib/contin,/home/hleal/papers/doblelim/bib/netwth}


\IEEEpeerreviewmaketitle


\end{document}


